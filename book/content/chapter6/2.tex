生活在现代社会的我们早已习惯基础设施的可靠运行——电力供应、清洁水源和卫生系统都已融入生活背景,无需考虑其背后的维护工作。而软件作为后来者,已悄然渗透到从支付到娱乐、从急救服务到交通出行的各个领域,其重要性却常被忽视。

内存安全的重要性不言而喻。轻则导致程序重启的轻微故障,重则成为黑客利用的漏洞或引发系统级故障。虽然不同软件对安全的要求各异(医疗系统与单机游戏自然不同),但任何软件都应重视这个问题。程序员有责任编写既功能完善又能保护用户安全的代码,甚至偶尔带来惊喜。我们编写的代码最终服务于人,而人永远是最重要的考量。尽管内存安全并非唯一难题,但直面它是进步的关键。

然而在众多现代服务中,软件虽带来便利却也潜藏危机:个人信息泄露、资金被盗、医院系统遭勒索软件攻击延误救治...软件无处不在,就必须做得更好。可我们程序员却常对这些问题视而不见,总以"软件本就复杂"、"用户自身不慎"、"没有无bug的程序"为借口。确实,软件日益复杂,技术迭代加速,半年前完好的代码可能如今就已过时——但这不能免除我们的责任。

航空业同样面临复杂系统与人为因素,却通过完善体系使飞行成为最安全的交通方式。反观软件行业,直到2024年2月26日白宫发布技术建议,要求国家安全相关应用采用内存安全语言(包括Java、C\#、Python、Rust,明确排除C/C++)时,多数人才如梦初醒(详见:\url{https://www.whitehouse.gov/oncd/briefing-room/2024/02/26/press-releasetechnical-report/})。

该报告起初引发惊讶、调侃与不安,但2024年7月19日CrowdStrike内存错误引发全球Windows系统内核恐慌时,其重要性终获印证:航班停飞、急救系统瘫痪、支付中断,数百万人生活受影响。这或许是大众首次真切意识到软件的重要性,也意味着软件安全将进入政策制定者的视野。