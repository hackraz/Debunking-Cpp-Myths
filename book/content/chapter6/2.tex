在现代社会中,我们早已习惯基础设施的稳定运行 --- 电力供应、清洁水源和卫生系统如同生活的背景音,无需我们操心其背后的维护与运转。而软件作为后起之秀,已悄然渗透进人类生活的方方面面:从支付与娱乐,到紧急医疗服务,再到交通出行。尽管如此,却常常忽视软件的重要性。

内存安全问题正是这一现象的典型体现。它不仅可能导致程序崩溃或需要重启的小故障,更可能成为黑客攻击的入口,甚至引发系统级灾难。虽然不同领域的软件对安全性的要求各不相同(例如医疗系统与单机游戏显然不可同日而语),但无论在哪一个领域,开发者都有责任编写功能完善、行为可靠且能保护用户安全的代码 --- 甚至偶尔还能带来惊喜。毕竟,我们所写的每一行代码,最终都是为了服务“人”本身。而人,永远是最重要的考量因素。

然而,在享受现代软件所带来的便利的同时,我们也必须面对它潜藏的风险:个人信息泄露、资金被盗、医院系统遭受勒索软件攻击导致救治延误……随着软件无处不在,它也必须做得更好。但许多开发者却往往选择视而不见,用“软件本就复杂”、“用户自己操作不当”或“没有绝对无 bug 的程序”作为借口来推卸责任。诚然,软件确实越来越复杂,技术更新也日益加速,半年前还运行良好的代码,如今可能已因环境变化而失效。但这些现实挑战,并不能成为我们逃避责任的理由。

航空业同样面临复杂系统与人为因素的挑战,但它通过建立完善的体系和严格的标准,使飞行成为当今最安全的交通方式之一。相比之下,软件行业直到近年来才开始真正重视安全性问题。2024年2月26日,白宫发布了一份技术建议报告,明确要求国家安全相关应用使用内存安全语言(包括 Java、C\#、Python 和 Rust),并将 C/C++ 明确排除在外(详见:\url{https://www.whitehouse.gov/oncd/briefing-room/2024/02/26/press-releasetechnical-report/})。这份报告的出台,让许多人第一次意识到内存安全问题的严重性。

起初,这份报告引发了惊讶、调侃与不安。然而仅仅五个月后,2024年7月19日,CrowdStrike 的一次内存错误便在全球范围内引发 Windows 系统内核恐慌,造成航班停飞、急救系统瘫痪、支付中断等严重后果,数百万人的生活因此受到影响。这次事件无疑是一记警钟,标志着软件安全问题不再只是开发者的内部议题,而是已经真实影响到社会运作的关键环节。

这也是为什么我们必须正视内存安全问题。它不仅是技术层面的挑战,更是关乎信任、责任与未来政策方向的重要议题。
