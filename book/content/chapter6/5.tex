
假设我们仅使用 STL 容器、避免裸指针,在必要时采用标准库提供的智能指针,并始终以内存安全为准则来设计类型 --- 这样就真的可以高枕无忧了吗?

C++ 标准委员会知名成员 Herb Sutter 在其 2024 年 3 月 11 日发表的博客《C++ 安全性的现状与思考》(\url{https://herbsutter.com/2024/03/11/safety-in-context/}) 中,深入探讨了这一问题。他的结论是:在默认情况下,C++ 代码仍然太容易引入安全隐患。文章指出,有四个关键领域亟需重点关注:类型系统、边界检查、初始化机制以及生命周期管理。尽管 C++20 已经引入了一些初步解决方案(如 std::span、std::string\_view、concepts 和范围感知机制),但语言层面仍缺乏一套“默认开启、按需关闭”的安全规则体系。

接下来,我们将结合具体实例,解析这些观点。

首当其冲的是C++20引入的std::span --- std::span 表示对一段连续内存(例如原生数组、std::array、带尺寸的指针、std::vector 或 std::string)的安全视图。它最大的优势在于能够自动推导出序列的长度,从而有效杜绝常见的“差一错误”(off-by-one errors)。这使得我们可以:安全地向函数传递集合的一部分,而无需担心越界访问;彻底禁用不安全的指针算术运算,转而使用 std::span 作为更安全的替代方案。

其次是 string\_view。std::string\_view 提供了一种对字符串内容的只读访问方式,极大地减少了因意外修改字符串内容而导致的安全隐患。此外,还消除了许多原本不安全的字符串操作模式,比如不必要的复制和格式错误。

第三是概念(concepts)。概念允许开发者为泛型函数或类施加清晰的约束条件,从而增强类型安全性。例如,可以明确要求某个泛型函数的参数类型必须支持加法和减法运算。虽然 concepts 的设计仍在持续演进(C++26 将带来更多改进),但它们已经能够帮助我们规避大量潜在的安全缺陷。

第四是边界感知的范围(bound-aware ranges)。C++20 中引入的范围(ranges)库,让开发者可以用更加简洁、函数式的方式处理集合数据。借助范围,开发者不再需要手动传递 begin() 和 end() 迭代器,因为范围本身就知道自己的边界,从而显著降低误用的风险。

这些现代 C++ 的改进措施(如果被正确使用)确实使 C++ 比 C++98 时代更加安全可靠。然而,语言的安全性仍有局限。

还记得之前那个访问未预留内存的 std::vector 索引,最终导致运行时内存错误的示例吗?让我们再来看一遍那段代码:

\begin{cpp}
int doSomeWork(int value1, int value2, int value3, int value4) {
  vector<int> values;
  values[0] = value1;
  values[1] = value2;
  values[3] = value3;
  values[4] = value4;
  return values[0] + values[1] + values[3] + values[4];
}
\end{cpp}

这段代码的问题在于,我们可以毫无阻碍地访问超出 vector 已分配大小的索引,同时跳过了对索引 2 的初始化。这正是未定义行为的典型表现。

为了解决此类问题,一个可行的方案是:

\begin{itemize}
\item 
启用编译器的安全模式标志

\item 
编译器在每次索引访问时自动生成边界检查,确保满足条件: \verb|0 <= index < collection.size()|

\item 
在运行时拦截越界访问,从而防止未定义行为的发生
\end{itemize}

这种编译选项可以在不修改现有代码的前提下直接应用,有效提升程序的健壮性,并帮助发现潜在的安全隐患。当然,一些开发者可能会出于对性能损耗的担忧而反对这一机制 --- 这正是此类检查应通过编译器标志进行控制的原因:理想情况下,默认启用,仅在性能敏感的场景中允许显式关闭。

由此可见,尽管现代 C++ 已经在内存安全方面取得了显著进步,但要真正实现全面的安全保障,语言和工具链仍有进一步改进的空间。