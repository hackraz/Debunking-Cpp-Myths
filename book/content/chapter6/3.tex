
在深入探讨传统C++与现代C++的内存安全问题前,让我们先明确其定义。援引白宫报告所述:"内存安全漏洞是指一类由于内存访问、写入、分配或释放操作违反预期方式而产生的漏洞(...)主要分为两大类:空间安全和时间安全。空间安全问题源于对变量和对象内存'正确'边界之外的访问;时间安全问题则发生在内存访问违反时序或状态时,例如访问已释放对象或出现意外交错的内存访问。"完整定义参见报告第7页:\url{https://www.whitehouse.gov/wp-content/uploads/2024/02/Final-ONCD-TechnicalReport.pdf}。

所有C++程序员都应对这两类问题耳熟能详。空间安全问题在使用原生数组时最为常见,例如下面这个创建数组、赋值后尝试越界读写元素的程序:

\begin{cpp}
int doSomeWork(int value1, int value2, int value3, int value4) {
  int array[3];
  array[0] = value1;
  array[1] = value2;
  array[3] = value3;
  array[4] = value4;

  return array[0] + array[1] + array[3] + array[4];
}
\end{cpp}

这段代码最令人意外的是它属于未定义行为(undefined behavior)——编译器可能以不同方式处理数组越界访问,从静默允许到报错皆有可能。更复杂的是,根据操作系统和运行环境,代码可能"正常"运行并覆盖未知内存区域。

\begin{myNotic}{Note}
对攻击者来说,这类代码就是金矿。为什么?因为虽然概率很小,但这个进程有可能在某个时间点与重要进程在内存中相邻存放。如果攻击者能向该函数传入特定值,并在恰当时机捕获进程,就可能实现:覆盖银行应用程序的身份验证检查代码/安装键盘记录器/向系统植入恶意软件。当然,这种攻击不能保证成功,但由于整个过程都是自动化的,黑客有的是时间尝试——他们只需要成功一次就够了。
\end{myNotic}

我在Ubuntu Linux系统上使用clang和g++编译器测试了这段代码。g++编译器即使开启所有警告选项也毫无提示地通过了编译,而clang则在编译时给出了数组越界访问的警告。实际运行程序时,系统显示"检测到堆栈破坏:程序已终止"。这说明运行时存在一定保护机制,但代码仍可能产生未知的副作用。

需要注意的是,这只是一个简单示例。如果在代码中创建数组后将其传递给多个函数,并通过复杂公式计算索引,恐怕没有编译器能识别这类越界访问。最终我们只能依赖测试和操作系统保护机制。

这是编程语言的缺陷吗?鲜为人知的是,那些被列为内存安全的语言同样可能写出这类代码。以C\#为例,它允许通过unsafe代码块使用指针和指针运算(尽管有限制)。关键区别在于:

\begin{itemize}
\item 
需要显式声明unsafe代码块

\item 
实现相同功能需要更多工作

\item 
不会产生C++那样的严重后果

\item 
危险代码因unsafe标记而显眼可见
\end{itemize}

C++的问题不在于允许这些操作,而在于默认情况下太容易犯错。既然谈到指针,请看下面这段通过void和指针运算越界访问int类型内存的代码:

\begin{cpp}
int pointerBounds() {
  int *aPointerToInt;
  void *aPointerToVoid;
  aPointerToVoid = new int();
  aPointerToInt = (int*)aPointerToVoid;
  *aPointerToInt = 234;
  aPointerToInt = (int*)((char*)aPointerToVoid + sizeof(int));
  *aPointerToInt = 2423;
  int value = *aPointerToInt;
  delete aPointerToVoid;
  return value;
}

TEST_CASE("try pointer bounds"){
  int result = pointerBounds();
  CHECK_EQ(2423, result);
}
\end{cpp}

我们再次遭遇未定义行为:通过指针算术运算得到的指针赋值操作,其具体行为由编译器决定。这次g++和clang都给出了警告——但仅针对删除void指针这一操作。两个编译器对我尝试读写越界内存区域的行为都毫无反应。更有趣的是,测试运行完全正常,函数返回了预期结果,皆大欢喜!甚至连操作系统都没对这荒谬操作提出抗议——但愿只是因为我没超出进程的内存分配范围。

但愿如此。

至此我们看到的都是空间内存安全问题,情况已经不容乐观。那么时间安全问题呢?任何使用过指针的人都深有体会:必须在指针不再需要时完成两件事——释放分配的内存并将其置为NULL。这两者都至关重要,因为忽略任一项都会引发时间安全问题:要么产生悬垂指针(仍能访问已释放内存区域),要么导致内存泄漏(指针未释放且可能已被重置,致使内存区域永久不可访问)。

请看下面这个示例函数:它初始化一个int指针并赋值,释放内存后,又返回该内存存储的值:

\begin{cpp}
int danglingPointer() {
  int *aPointerToInt = new int(234);
  delete aPointerToInt;
  return *aPointerToInt;
}

TEST_CASE("Try dangling pointer"){
  int result = danglingPointer();
  CHECK_EQ(234, result);
}
\end{cpp}

这段代码再次顺利通过了编译——无论是g++还是clang都没有发出任何警告。程序也能运行,但测试最终失败:因为此时该内存地址存储的值已非预期结果。

\begin{shell}
test.cpp:8: ERROR: CHECK_EQ( 234, result ) is NOT correct!
  values: CHECK_EQ( 234, 721392248 )
\end{shell}

每次调用该函数时,该内存地址存储的值都会发生变化,可能产生如下不同的结果:

\begin{shell}
test.cpp:8: ERROR: CHECK_EQ( 234, result ) is NOT correct!
  values: CHECK_EQ( 234, 1757279720 )
test.cpp:8: ERROR: CHECK_EQ( 234, result ) is NOT correct!
  values: CHECK_EQ( 234, -1936531037 )
\end{shell}

在后续代码中使用这个值进行计算并返回异常结果,简直出乎意料地简单。如果你对程序执行的运算稍有了解,并能反复输入特定值,这甚至可能成为探查内存区域内容的手段。

时间安全问题更为棘手——在庞大的代码迷宫中追踪指针生命周期,远比确保不越界访问困难得多。是的,很不幸,内存问题确实是C++中的重大隐患。

不过您可能已经注意到,前述所有示例都采用了传统C++风格:使用原生数组、裸指针和指针运算。正因如此,在现代C++中应当极其谨慎地使用这些特性。虽然不能说完全禁用(毕竟某些特定场景如内存优化或底层编程仍需它们),但通常可以划定明确边界将其隔离在现代C++的安全区域之外。

那么,现代C++是否已解决所有这些问题?