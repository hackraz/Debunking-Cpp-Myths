
我们重新审视前面的示例,但按照现代C++的建议,将原生数组和原生指针替换为对应的STL容器。

首先来看数组越界的例子。我们只需将原生数组替换为vector<int>实例,就能得到以下函数:

\begin{cpp}
int doSomeWork(int value1, int value2, int value3, int value4) {
  vector<int> values;
  values[0] = value1;
  values[1] = value2;
  values[3] = value3;
  values[4] = value4;
  return values[0] + values[1] + values[3] + values[4];
}

TEST_CASE("try vector bounds"){
  int result = doSomeWork(1, 234, 543, 23423);
  CHECK_EQ(1 + 234 + 543 + 23423, result);
}
\end{cpp}

遗憾的是,运行这个示例的结果并不理想。无论是g++还是clang都没有报错,运行测试时得到以下结果:

\begin{shell}
TEST CASE: try vector bounds
test.cpp:5: FATAL ERROR: test case CRASHED: SIGSEGV - Segmentation violation signal
\end{shell}

难道std::vector<>也不安全吗?实际上我们仍需注意其内存分配问题。我们有几种选择:正确初始化容器、使用提供的追加元素方法,或者预先预留特定数量的内存空间。前两种是我通常推荐的方式,因为它们更不容易出问题。但即便采用第三种方式,测试也能通过:

\begin{cpp}
int doSomeWork(int value1, int value2, int value3, int value4) {
  vector<int> values;
  values.reserve(5);
  values[0] = value1;
  values[1] = value2;
  values[3] = value3;
  values[4] = value4;
  return values[0] + values[1] + values[3] + values[4];
}
\end{cpp}

一个令人惊喜的发现是std::vector的行为(至少在g++中)——调用reserve后,我尝试访问未赋值的values[2],得到的值是0。这比直接访问该内存块残留的旧值要好得多,我猜这是std::vector默认分配器的特性。这种差异源于operator[]的未定义行为,而使用vector::at()方法可以避免该问题。不过我们仍需谨慎行事——即便使用现代STL,仍然可能写出存在内存问题的代码。当然,如果我们不耍小聪明,老老实实采用简单方法,问题就能彻底避免。比如使用初始化列表语法时,向量会根据传入的数据自动创建,完全不需要手动计数:

\begin{cpp}
int doSomeWork(int value1, int value2, int value3, int value4) {
  vector<int> values{value1, value2, 0, value3, value4};
  return values[0] + values[1] + values[3] + values[4];
}
\end{cpp}

当然,这种语法要求我们一次性添加所有元素,而非部分元素,从而有效规避"差一错误"。另一种方式是逐个添加元素:

\begin{cpp}
int doSomeWork(int value1, int value2, int value3, int value4) {
  vector<int> values;
  values.push_back(value1);
  values.push_back(value2);
  values.push_back(0);
  values.push_back(value3);
  values.push_back(value4);
  return values[0] + values[1] + values[3] + values[4];
}
\end{cpp}

正如预期,这个版本同样完美运行。启示在于:使用最朴素的写法,99\%的情况下都能获得预期行为。这是所有编程语言的通用准则,对C++而言更是如此。

现在让我们重新审视那个使用指针运算和void*越界访问内存的例子。其原始代码如下:

\begin{cpp}
int pointerBounds() {
  int *aPointerToInt;
  void *aPointerToVoid;
  aPointerToVoid = new int();
  aPointerToInt = (int*)aPointerToVoid;
  *aPointerToInt = 234;
  aPointerToInt = (int*)((char*)aPointerToVoid + sizeof(int));
  *aPointerToInt = 2423;

  int value = *aPointerToInt;
  delete aPointerToVoid;

  return value;
}
\end{cpp}

我已尽力将这段代码改用std::unique\_ptr或std::shared\_ptr实现,虽然可行但极其繁琐。首要难题在于处理所有的指针类型转换——无法直接将std::unique\_ptr<int>转为std::unique\_ptr<char>,唯一方法是提取原始值重新构造unique\_ptr<>实例。

第二个难点是void*无法直接对应到std::shared\_ptr<void>:虽然可以通过手动分配内存并传入自定义删除器来实现,但过程复杂。因此,即使用现代STL勉强能写出类似功能的代码,其实现成本之高也足以让开发者除非有特殊需求,否则宁愿选择安全写法。

悬垂指针的例子同样棘手:

\begin{cpp}
int danglingPointer() {
  int *aPointerToInt = new int(234);

  delete aPointerToInt;

  return *aPointerToInt;
}
\end{cpp}

智能指针无法在返回数值的同时被主动释放。虽然可以通过unique\_ptr::reset重新分配内存,但这又绕回了原生指针。最简洁的纯智能指针实现如下:

\begin{cpp}
int danglingPointer() {
  unique_ptr<int> aPointerToInt = make_unique<int>(234);
  return *aPointerToInt;
}
\end{cpp}

唯有这样才能完全符合预期:正确返回值的同时自动释放内存,从根本上杜绝悬垂指针!当然,如果手动分配内存并为std::unique\_ptr<>设置空删除器,仍可能制造悬垂指针——但绝大多数场景都没有理由这么做。若需内存块多归属管理,还可选用std::shared\_ptr<>,至此常见需求均已覆盖。

由此我们得出结论:现代C++通过默认机制大幅提升了安全性,但仍存在局限性——这正是接下来要探讨的重点。