我的第一台电脑是罗马尼亚产的HC-90,这是ZX-80的克隆机型。我拥有过两个版本:第一个版本需要用磁带播放器加载程序。尽管操作不便,但它比当时的主要竞争对手CHIP电脑(另一款罗马尼亚产的ZX-80克隆机)有个显著优势——CHIP电脑需要磁带加载操作系统,而HC-90的EPROM存储器足以直接启动BASIC解释器。我拥有的第二个版本进步很大:配备了5英寸软盘驱动器,这意味着程序加载速度大幅提升。

在这两个版本上,BASIC解释器都是人机交互的主要界面。由于除了游戏外可用程序很少,我的高中时光大多在编写BASIC程序和玩游戏度过。渐渐地,BASIC已不能满足我的需求。虽然尝试过图形和声音编程,但运行速度极其缓慢,这促使我学习ZX-80汇编语言——这堪称一场冒险。汇编编程极易出错,一个失误就会导致系统重启和工作丢失。虽然这种编程方式难称高效,却让我格外珍惜当今的编程环境:能在电脑上编译运行测试,还能将修改保存到版本控制系统。

当时我渴望提升图形和声音的响应速度,却没意识到存在根本性限制:单核CPU(按现今说法)意味着图形渲染、声音处理和逻辑运算必须串行执行。CPU可以先处理声音指令,再显示图形,最后进行计算。由于指令执行与实际声画输出存在微小延迟,这些任务看似并行,实则只是并发。当系统满负荷运行时,就能观察到声画不同步的现象。

如果当时有多处理器或多核环境会怎样?我本可以将不同任务分配到独立处理器上。一个高效的调度器能将这些任务真正并行执行以充分利用闲置CPU资源。只要任务定义明确,我们就能充分挖掘多核潜力,更快获得结果——这才是真正的并行计算。

来自Haskell社区的定义差异颇具启发性(参见\url{https://wiki.haskell.org/Parallelism_vs._Concurrency})。他们严格区分并行函数式程序与并发函数式程序:两者都采用不可变性原则,但并行程序通过多核加速执行且保持确定性(无论串行还是并行运行,程序语义不变);而并发程序运行的非确定性线程各自执行I/O操作,操作顺序不可预知。

遗憾的是,如同软件开发中的常见现象,这些术语的含义常被随意解读。有人坚称并发与并行截然不同。我在StackOverflow上就看到有观点认为并发是并行的超集,因为并发泛指管理多线程的各种方法——这也确实是某些计算机教材的立场。

为明确概念,我选择最契合我编程启蒙时期的定义:并发是指多个操作看似同时运行,而并行是指它们确实同时运行。这种看似简单的差异会导致程序设计意图的根本不同。设计并行程序时,我们定义可并行执行的操作并确定其顺序,通过分解大任务来榨取CPU时间;设计并发程序时,我们则通过将长任务填入CPU空闲时段来优化响应时间。

这两种编程模型各有其独特挑战。现在让我们回顾使用它们时的常见问题。