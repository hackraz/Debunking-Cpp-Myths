我的第一台电脑是罗马宁亚制造的 HC-90,这是 ZX-80 的克隆机型。我拥有过两个版本:第一个版本需要用磁带播放器加载程序。尽管操作不便,但它相比当时的主要竞争对手 CHIP 电脑(另一款罗马尼亚产的 ZX-80 克隆机)有一个显著优势--- CHIP 需要从磁带加载操作系统,而 HC-90 的 EPROM 足以直接启动 BASIC 解释器。我拥有的第二个版本进步很大:配备了 5 英寸软盘驱动器,这意味着程序加载速度大幅提升,也标志着我开始真正接触更复杂的编程任务。

在这两个版本上,BASIC 解释器都是人机交互的主要界面。由于除了游戏之外几乎没有其他可用程序,我的高中时光大多是在编写 BASIC 程序和玩游戏中度过的。渐渐地,BASIC 已不能满足我的需求。虽然我也尝试进行图形和声音编程,但运行速度极其缓慢,这促使我学习 ZX-80 的汇编语言---这堪称一场冒险。汇编编程极易出错,一个小小的失误就可能导致系统重启并丢失所有未保存的工作。虽然这种编程方式难称高效,但它让我格外珍惜如今的编程环境:能在电脑上编译、运行、测试代码,并将修改保存到版本控制系统中。

当时我渴望提升图形和声音的响应速度,却没意识到存在根本性限制:单核 CPU(按现今说法)意味着图形渲染、声音处理和逻辑运算必须串行执行。CPU 可能先处理声音指令,再显示图形,最后进行计算。由于这些任务在执行与实际输出之间存在微小延迟,它们看似“并行”,实则只是并发。当系统满负荷运行时,就能观察到声画不同步的现象。

如果当年我能使用多处理器或多核环境会怎样?我可以将不同的任务分配到独立的处理器上。一个高效的调度器可以真正并行执行这些任务,从而充分利用闲置的 CPU 资源。只要任务定义明确,就能充分挖掘多核潜力,更快获得结果 --- 这才是真正的并行计算。

来自 Haskell 社区的一个定义差异颇具启发性(参见\url{https://wiki.haskell.org/Parallelism_vs._Concurrency})。他们严格区分并行函数式程序与并发函数式程序:两者都采用不可变性原则,但并行程序通过多核加速执行,同时保持确定性(无论串行还是并行运行,程序语义不变);并发程序则涉及多个非确定性线程,各自执行 I/O 操作,操作顺序不可预知。

遗憾的是,如同软件开发中的常见现象,这些术语的含义常被随意解读。有人坚称并发与并行截然不同。我在 StackOverflow 上就看到有观点认为并发是并行的超集,因为并发泛指管理多线程的各种方法 --- 这也确实是某些计算机教材所采用的立场。

为明确概念,我选择最契合我编程启蒙时期的定义:并发是指多个操作看似同时运行,而并行是指它们确实同时运行。这种看似简单的差异会导致程序设计意图的根本不同。设计并行程序时,定义可并行执行的操作并确定其顺序,通过分解大任务来榨取 CPU 时间;设计并发程序时,则通过将长任务填入 CPU 空闲时段来优化响应时间。

这两种编程模型各有其独特挑战。现在,回顾使用它们时的常见问题。