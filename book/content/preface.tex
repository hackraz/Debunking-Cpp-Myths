将 C++ 想象成一门神秘而古老的编程语言,它承袭自最初的编程之源,在底层魔法的烈焰中锻造而成,又以高层抽象咒语般的精准加以淬炼。C++ 诞生于一种双重需求:既要掌控机器本身,又要提供高层次的抽象表达。它是那些试图在原始机器与高级结构之间架起桥梁、同时又能享受现代工具便利的程序员手中最锋利的武器。

而这本书也绝非你曾见过的任何一本。作者们踏上了一段大胆的探索之旅,深入 C++ 迷宫般复杂的深处,只为揭示这门语言的真正面貌。他们以勇气与精准的洞察力,试图揭开长久以来笼罩在这门传奇语言之上的种种迷雾与误解,直面它的辉煌之处,也不避讳它所被认为的缺陷。

以开放的心态来阅读这本书吧,因为它将带你踏上一段非同寻常的旅程——这是一段不适合胆怯者的旅程。作者们毫不犹豫地深入探讨了 C++ 那复杂的名声:直面它臭名昭著的指针和繁琐的内存管理,甚至潜入底层 C++ 的世界,在那里汇编语言主宰一切,而指针不过只是数字而已。我们将审视现实中各种不同的 C++,观察其生态系统,了解在今天该如何学习它,并辨识出那些你最好忘记的做法。在每一章中,我们都将层层揭开,探索这些强大机制背后的逻辑与优雅。书中穿插着关于 C++ 传奇人物的故事,以及偶尔略显“独特”的幽默感,目的是让你保持兴趣,同时引导你穿越 C++ 的崇高与荒诞。你将会遇到一些前所未有的糟糕代码——它们是特意呈现出来的,为的是告诉你什么不该做,同时也揭示出 C++ 真正的潜力。这本书的目的不只是教你编程,更是要透过 C++ 的优点与陷阱,展现它的灵魂所在。

\mySubsectionNoFile{}{适读人群}

这本书以打破迷思、略带幽默的方式呈现,非常适合那些已经具备 C++ 基础、并希望进一步探索其细节与奥秘的程序员。同时,它也适合充满好奇心的学习者和计算机科学专业的学生——他们被 C++ 强大而复杂的声誉所吸引,渴望了解这门语言背后的逻辑与魅力。

本书的读者还包括那些欣赏编程艺术性与哲学内涵的人:他们不只是想“使用”C++,更希望理解它为何如此运作,以及它最著名(或臭名昭著)特性的背后故事。对于这些人来说,编程不仅是一项技能,更是一门技艺,一门由历史、个性乃至些许传奇共同塑造的学问。

\mySubsectionNoFile{}{本书内容}

第1章《C++很难学》 探讨了这一现象背后的原因:究竟是语言本身的问题,还是教学方法的问题?我们是否应该从指针和内存管理等底层特性开始学习?或者,也许我们应该先从实际可运行的示例或面向对象编程(OOP)功能入手?此外,每位 C++ 程序员都需要掌握完全相同的 C++ 吗?本章将围绕不同的语言学习方法展开讨论,重点关注 C++,并最终判断在今天——只要采用合适的方法——C++ 是否仍然难学。

【作者:Alex】

第2章《每个 C++ 程序都符合标准》 回答了标题所提出的问题。在一个理想世界中,或许确实如此!实际上,每个 C++ 程序都应该符合标准。然而正如我们在本章中所发现的那样,当程序稍有偏离,比如使用了一些晦涩的编译器扩展、尝试接触未定义行为,或是依赖特定平台的独特特性时,你可能瞬间就会陷入一堆只有古代神秘学者才能解读的错误之中。所以,是的,每个 C++ 程序都是“合规”的……直到它不再合规为止!

【作者:Ferenc】

第3章《只有一种 C++,而且它是面向对象的》 探讨了不同的代码组织范式,包括函数式编程、元编程,以及较少为人知的极端多态性等内容。

【作者:Alex】

第4章《main() 函数是你的应用程序入口点》 正如标题所示,探讨了 main() 函数的作用。实际上,正如本章所述,main() 函数就像我们程序的前门:一切由此开始,但如果你掀开它的面纱,往往会发现其背后是一张复杂的依赖网络、库文件和操作系统相关的系统调用。要真正走到这个入口,更像是穿越迷宫,而非沿着一条直路前行。

【作者:Ferenc】

第5章《C++ 类中必须有序》 探讨了一个基本事实:没错,在一个 C++ 类中,秩序是必不可少的,因为缺乏秩序就会引发问题!方法、数据成员、构造函数,每一个都有其应有的位置!是的,C++ 提供了一定的灵活性,但结构不能被忽视。如果不尊重类成员的排列顺序,整个类就可能崩溃!自由过多,未定义行为随之而来,错误、漏洞、崩溃也随之而至!混乱,是 C++ 所不能容忍的。唯有遵循顺序,方能实现和谐统一!本章将介绍一些最重要的规则,说明 C++ 中哪些概念的排列顺序是有明确规定的,又有哪些虽然没有明确规定,却依然至关重要。

【作者:Ferenc】

第6章《C++ 并非内存安全的语言》 探讨了 C++ 中内存管理所面临的挑战,现代语言特性的承诺及其局限性,并将其置于公众对软件可靠性日益关注的大背景下进行分析。

【作者:Alex】

第7章《C++ 中并没有简单的并发与并行方式》 探讨了并发与并行的需求,现代 C++ 是如何应对这些需求的,以及 actor 模型如何帮助你在产品中设计并行机制。

【作者:Alex】

第8章《最快的 C++ 代码是内联汇编》 讲述了一个我们三十年前就被教导的事实。诚然,汇编语言确实提供了底层控制能力,但现代编译器已经高度优化,往往能够生成比手动编写汇编更高效的代码,这将在本章中得到演示。的确,在某些情况下,内联汇编可以提升性能,但它牺牲了可读性和可移植性。因此,请谨慎使用,并仅在绝对必要时才使用它。

【作者:Ferenc】

第9章《C++ 是美丽的》 坚定地认为:C++ 确实是美丽的。试问,还有哪种语言能如此优雅地被尖括号、分号、大括号和句点所缠绕?它是一场关键词、模板、古老宏定义和重载运算符交织而成的诗意舞蹈,所有这些元素巧妙排列,甚至能让最资深的程序员对自己的人生选择产生怀疑。诚然,正如本章将展示的那样,C++ 的语法正是美的化身——如果“美”意味着一个包裹在谜团中的谜题,再加上一点预处理器对不可预处理内容的再次混淆的话。

【作者:Ferenc】

第10章《C++ 缺乏现代编程所需的库》 探讨了现代 C++ 对库的需求与现状,包括包管理的挑战、为特定目标版本和架构寻找合适库的困难,以及日益严重的供应链攻击问题。

【作者:Alex】

第11章《C++ 向后兼容……甚至是兼容 C》 探讨了向后兼容这一特性。正如本章所述,C++ 继承了一个“家族传家宝”:一堆混乱的全局变量、危险的指针,以及未定义行为。C++ 忠实地保留了这些遗留特征,让这两种语言得以在一个笨拙却出奇可行的方式中共存。的确,这种兼容性令人振奋:谁不想把几十年前的 C 代码和现代 C++ 混合在一起呢?哪怕不是那么“现代”的 C++?说到底,传统很重要嘛,我们总得为了生计而努力攀登!

【作者:Ferenc】

第12章《Rust 将取代 C++》 探讨了为何我们需要这么多编程语言,Rust 是如何融入整个生态系统的,它擅长的领域,C++ 又作出了怎样的回应,以及在什么条件下 Rust 有可能取代 C++。

【作者:Alex】

\mySubsectionNoFile{}{如何阅读}

本书的理想读者是具备一定基础到经验丰富的 C++ 开发人员,以及学术型学习者——他们已经掌握了扎实的编程基础知识,并渴望进一步深入探索 C++ 的复杂细节。

在实际项目中使用 C++ 的专业人士、希望通过汇编语言或高级编译技术来优化性能的开发者,以及那些欣赏这门语言独特个性与复杂性的爱好者,可能会从本书中获得阅读的乐趣。

对于希望系统了解 C++ 的计算机科学学生、追求展示最新现代 C++ 技术的学者,或者正在学习这门语言的程序员,请注意:本书并不涵盖 C++ 的入门知识,也不会讲解“如何学习 C++”。这个任务更适合其他一些书籍来完成,例如由 C++ 之父 Bjarne Stroustrup 所著的经典教材《Programming — Principles and Practice Using C++(第3版)》,当然,也可以选择任何适合你风格的入门书籍。

而对于那些经验丰富的 C++ 开发者,希望借此概览最新 C++ 标准的语言专家(language lawyers),或是缺乏幽默感的人,又或者你是为了解决某个迫在眉睫的问题而翻开这本书的读者来说……那么你很可能会觉得这本书并不合你的胃口。因为这本书可能根本不会回答你提出的问题,甚至它本身可能也没有答案。相反,你可能会发现,读完之后反而比以前有了更多的疑问。如果你属于这类读者,我建议你直接去阅读 C++ 标准文档——所有问题的答案都在那里。你已经被警告过了。

% Please add the following required packages to your document preamble:
% \usepackage{longtable}
% Note: It may be necessary to compile the document several times to get a multi-page table to line up properly
\begin{longtable}{|p{7.5cm}|p{7.5cm}|}
\hline
\textbf{书中涉及的软件/硬件}          & \textbf{操作系统要求}               \\ \hline
\endfirsthead
%
\multicolumn{2}{c}%
{{\bfseries Table \thetable\ continued from previous page}} \\
\endhead
%
各种c++编译器,它们与2025相关或不相关 & Windows, macOS, Linux,或者根本没有操作系统 \\ \hline
\end{longtable}

\textbf{如果正在使用本书的数字版本,我们建议您自己输入代码或通过GitHub存储库访问代码(下一节提供链接),将避免复制和粘贴代码。}

\mySubsectionNoFile{}{下载源码}

您可以从 GitHub 下载本书的示例代码文件 \url{https://github.com/PacktPublishing/Debunking-CPP-Myths}。如果代码有更新,它将在 GitHub 存储库中更新。

还有丰富的书籍和视频目录中的其他代码包,可在 \url{https://github.com/PacktPublishing/} 上找到。快来看看吧!







