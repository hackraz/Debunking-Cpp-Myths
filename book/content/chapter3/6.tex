
几年前,我曾带领一个团队使用一种名为 Groovy 的语言和其配套框架 Grails 开发了多个 Web 应用程序。Groovy 是一门可选类型化 和 动态类型 的语言:允许运行时决定类型,也支持你为编译器提供类型提示。此外,Groovy 也可以静态编译,由于它构建在 JVM 上,最终会生成 Java 字节码。

在之前的 Web 项目中,我注意到类型在系统的边界处非常有用,例如:用于验证请求参数、与数据库交互或其他 I/O 操作。然而,在 Web 应用的核心逻辑中使用类型往往会带来不必要的复杂性。我们经常需要修改现有代码,或者编写适配层,以适应用户对已有功能提出的新用途。

Web 应用的用户常常会发现某个功能在一个场景下很有用,并希望它能在其他上下文或处理不同类型的数据时也能工作。因此,从一开始就决定:我们只在请求验证和与外部系统交互时使用类型,而在核心业务逻辑中不强制使用类型。

我们的策略始终围绕着一套稳健的自动化测试机制展开,确保所有代码都通过测试来验证其行为的正确性。我原本预计,缺乏类型检查会需要编写更多测试,但实际结果却出乎意料:测试数量并没有显著增加,但代码量明显减少了。

更有趣的是,由于没有类型辅助理解变量和函数的行为,我们需要更加重视命名 --- 因为对于开发者来说,名字成了唯一能帮助他们理解意图的线索。

直到今天,这种风格仍然是我最喜欢的编程方式。我希望按照自己的意愿、尽可能富有表现力地编写代码,然后让编译器去处理类型相关的细节。你可以将这种方法看作是极端多态:只要传入的对象拥有所需的方法,不管它的具体类型是什么,代码都应该能够正常运行。

这种风格并不适合所有人,它只有在配合特定的设计经验与实践(如良好的命名、清晰的接口设计、完善的测试覆盖等)时才真正有效。但它确实是一种值得尝试的编程范式。不过,迈出第一步往往是最难的 --- 需要放下对编译器行为的严格控制欲。对于习惯了细节导向的 C++ 开发者来说,这可能是一件颇具挑战的事情。

那么,如何在 C++ 中实现类似的风格呢?幸运的是,从 C++11 开始引入了 auto 关键字,并在后续标准中逐步增强了其能力。虽然 C++ 并不像 Groovy 那样具备真正的动态类型特性,但可以通过模板和泛型编程来模拟这种行为。

先用一个极具多态性的例子让你开开眼界:

\begin{cpp}
auto identity(auto value){ return value;}

TEST_CASE("Identity"){
  CHECK_EQ(1, identity(1));
  CHECK_EQ("asdfasdf", identity("asdfasdf"));
  CHECK_EQ(vector{1, 2, 3}, identity(vector{1, 2, 3}));
}
\end{cpp}

无论你传入什么类型的值,这个函数都能完美运行。是不是简洁又强大?想象一下,如果你的系统核心由一系列这样的函数组成,无需修改就能直接复用 --- 对我来说,这就是理想的编程环境。当然,现实比这要复杂得多,程序通常需要的功能远不止恒等函数那么简单。

再来看一个稍微复杂的例子:判断一个字符串是否是回文,也就是正着读和反着读都一样的字符串。一个典型的 C++ 实现如下:

\begin{cpp}
bool isStringPalindrome(std::string value){
  std::vector<char> characters(value.begin(), value.end());
  std::vector<char> reversedCharacters;
  std::reverse_copy(characters.begin(), characters.end(), std::back_
  insert_iterator(reversedCharacters));
  return characters == reversedCharacters;
}

TEST_CASE("Palindrome"){
  CHECK(isStringPalindrome("asddsa"));
  CHECK(isStringPalindrome("12321"));
  CHECK_FALSE(isStringPalindrome("123123"));
  CHECK_FALSE(isStringPalindrome("asd"));
}
\end{cpp}

如果我们尝试让这段代码对类型“不再敏感”,会发生什么呢?可以将参数类型改为 auto,并利用现代 C++ 的 Ranges 库来实现通用的反转与比较操作。最终得到的函数不仅适用于字符串,还可以处理诸如 vector<string> 或定义在枚举中的词法单元等结构。

来看看这种极端多态的实际效果:

\begin{cpp}
bool isPalindrome(auto value){
  auto tokens = value | std::views::all;
  auto reversedTokens = value | std::views::reverse;
  return std::equal(tokens.begin(), tokens.end(), reversedTokens.begin());
};

enum Token{
  X, Y
};

TEST_CASE("Extreme polymorphic palindrome"){
  CHECK(isPalindrome(string("asddsa")));
  CHECK(isPalindrome(vector<string>{"asd", "dsa", "dsa", "asd"}));
  CHECK(isPalindrome(vector<Token>{Token::X, Token::Y, Token::Y, Token::X}));
}
\end{cpp}

也许你现在已经开始理解为什么我对这种编程风格如此着迷了。当我们忽略类型,或者让函数具备极高的多态性,就能写出适用于未来各种情况的代码,而无需频繁修改。

但这当然也有代价:此时代码的约束条件隐藏在推导出的类型中,函数参数和命名变得至关重要。例如,传入一个整数给 isPalindrome 函数,不会看到一个明确的错误提示(比如“类型不匹配”),而是面对一段冗长且难以理解的编译错误信息。

以下是我在电脑上使用 g++ 编译器尝试传入整数时,所生成的部分错误输出:

\begin{shell}
In file included from testPalindrome.cpp:3:
Palindrome.h: In instantiation of 'bool isPalindrome(auto:21)
[with auto:21 = int]':
testPalindrome.cpp:30:2: required from here
Palindrome.h:14:29: error: no match for 'operator|' (operand types are 'int' and 'const std::ranges::views::_All')
14 | auto tokens = value | std::views::all;
   | ~~~~~~^~~~~~~~~~~~~~~~~
\end{shell}

现在问题来了:你更喜欢哪种方式?强类型带来的安全性与清晰性,还是极端多态所带来的灵活性与表达力?每种方式都有其权衡与适用场景。关键在于你如何看待代码的可维护性、扩展性以及开发效率之间的平衡。





















