

在软件开发中,最大的挑战之一就是避免 bug。这个问题如此普遍,以至于我们专门用“bug”这个词来形容它,仿佛代码出错是一种不可避免的现象。但事实上,大多数所谓的“bug”本质上只是开发者犯下的错误(mistakes)。

既然我们拥有编译器,那为什么不利用它来施加足够的限制,使得一旦出现潜在问题,编译器就能立即通知我们呢?理论上我们完全可以做到这一点,只不过这需要付出一定的设计与实现成本。

在前一节中,我们讨论了模板元编程(template metaprogramming),但我们忽略了一个关键特性:模板元编程是图灵完备的(Turing complete)。这意味着,任何可以用常规方式写出的程序,也可以通过模板元编程在编译时完成。

这一思想非常强大,也在多个领域被广泛探讨。如果你想尝试一门完全围绕这一理念构建的编程语言,不妨试试 Idris(\url{https://www.idris-lang.org/})。许多开发者可能也熟悉 Haskell 中在编译期进行验证的能力。而我第一次接触到这个概念,是在 Andrei Alexandrescu 于 2001 年出版的开创性著作《Modern C++ Design: Generic Programming and Design Patterns Applied》中。

让我们从一个简单的问题入手。“原始类型痴迷” 是导致 bug 和代码异味的一个常见来源。它指的是过度依赖原始类型(如 int 或 double)来表示本应更复杂的语义信息。例如,将长度、金额、温度或重量等值简单地表示为数字,却忽略了它们所代表的单位和含义。

与其这样,不如为这些概念定义特定的类型。以“金额”为例,它可以是一个包含上下文精度控制的值(比如会计系统中常用的七位小数)以及货币种类的封装类型。即使当前系统只处理一种货币,这种做法仍然非常有价值——因为你几乎可以肯定一件事:随着功能不断增加,原本只支持单一货币的系统,最终很可能需要支持多种货币。

与“原始类型痴迷”密切相关的一个问题是:如何对原始类型的取值范围进行约束。例如,一个表示一天中小时数的值,不应该只是一个普通的 unsigned int,而必须在 0 到 23 的范围内(这里我们假设使用的是 24 小时制)。如果我们能告诉编译器:“任何不在 0 到 23 范围内的值都不能作为合法的小时数”,并在传入非法值(如 27)时给出明确的错误提示,那就再好不过了。

在这种情况下,由于取值有限,使用 enum 可能是一个可行的解决方案。但为了说明运行时和编译时两种不同策略的设计思路,我们先来看一个运行时的实现方式。我们可以创建一个名为 Hour 的类,在构造函数中对输入值进行合法性检查,并在非法时抛出异常:

\begin{cpp}
class Hour{
private:
  int theValue = 0;
  void setValue(int candidateValue) {
    if(candidateValue >= 0 && candidateValue <= 23){
      theValue = candidateValue;
    }
    else{
      throw std::out_of_range("Value out of range");
    } 
  }

public:
  Hour(int theValue){
    setValue(theValue);
  }

  int value() const {
    return theValue;
  }
};

TEST_CASE("Valid hour"){
  Hour hour(10);
  CHECK_EQ(10, hour.value());
}

TEST_CASE("Invalid hour"){
  CHECK_THROWS(Hour(30));
}
\end{cpp}

么,如果我们希望将这种检查提前到编译时又该怎么做呢?这时就轮到 constexpr 大显身手了。我们可以结合 static\_assert 来在编译期验证数值是否合法:

\begin{cpp}
template <int Min, int Max>
class RangedInteger{
private:
  int theValue;
  constexpr RangedInteger(int theValue) : theValue(theValue) {}
public:
  template <int CandidateValue>
  static constexpr RangedInteger make() {
    static_assert(CandidateValue >= Min && CandidateValue <= Max, "Value out of range.");
    return CandidateValue;
  }

  constexpr int value() const {
    return theValue;
  }
};

using Hour = RangedInteger<0, 23>
\end{cpp}

在上面的实现中,下面这段测试代码可以完美运行:

\begin{cpp}
TEST_CASE("Valid hour"){
  constexpr Hour h = Hour::make<10>();

  CHECK_EQ(10, h.value());
}
\end{cpp}

但如果尝试传递一个超出范围的值,就会触发编译错误:

\begin{shell}
TEST_CASE("Invalid hour"){
  constexpr Hour h2 = Hour::make<30>();
}
Hour.h: In instantiation of 'static constexpr RangedInteger<Min, Max>
RangedInteger<Min, Max>::make() [with int CandidateValue = 30; int Min
= 0; int Max = 23]':
Hour.h:11:87: error: static assertion failed: Value out of range.
   11 | static_assert(CandidateValue >= Min && CandidateValue <= Max, "Value out of range.");
      |                                        ~~~~~~~~~~~~~~~^~~~~~
Hour.h:11:87: note: '(30 <= 23)' evaluates to false
\end{shell}

这条错误信息清楚地告诉我们:30 不是一个合法的小时值,而这正是我们想要的效果!

这只是众多技巧中的一种,适用于希望编写出在编译时就能证明计算合法有效的 C++ 程序。模板元编程是图灵完备的,理论上可以在编译时实现任何运行时能完成的任务。当然,这其中也存在权衡。

需要注意的是,这里的 Hour 值必须是 constexpr 的,会直接嵌入到可执行文件中。这是有意为之的,要将类型约束发挥到极致,唯一的方法就是将其直接编译进程序单元中。

在实践中,我发现这类技术虽然强大,但也容易让代码变得极其难以理解和维护。修改这类代码需要极高的纪律性,因为哪怕是一点点改动,也可能引入新的 bug --- 这些 bug 本来可以通过强类型机制避免。基本原则始终是:多用扩展的方式添加新功能,少去修改已有代码,除非是为了修复问题。

目前我们的示例还算简洁清晰,但随着类型抽象层次的加深,六个月后再回看当初的设计思路时,你可能会发现很难还原其背后的逻辑。从积极的一面来看,这种技术非常适合用于开发专注于特定领域的类库。

虽然我认为这种技术非常有趣,但在实际项目中我更倾向于保留更大的自由度。我会采用自己的一套方法:测试驱动开发(TDD)、持续重构、极致关注命名质量以及追求简单设计。我更希望有一种方式,让我写出我想写的代码,而让编译器去处理那些繁琐的细节。

正因如此,接下来我要介绍的最后一种编程范式,会尽可能地淡化类型的影响,从而提供另一种视角来看待程序设计与实现。