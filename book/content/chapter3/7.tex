在本章中,我们看到了 C++ 是一门支持多种编程范式的语言,开发者可以根据需求灵活选择不同的风格进行编程。我们简要探讨了几种不同的编程方式:函数式编程、元编程、借助强类型系统实现编译时验证,以及追求极致灵活性的极端多态性。

这些方法,连同传统的面向对象编程和结构化编程,各自适用于不同的上下文和场景,在构建类库或特定应用程序时都展现出其独特价值。对于那些渴望深入理解编程本质、不断探索技术边界的开发者来说,每一种范式都能带来新的启发与收获。

它们各有优劣,适用于不同类型的项目和开发目标。理解这些权衡不仅有助于写出更清晰、更高效的代码,也能帮助我们更好地理解软件设计的本质。

C++ 开发者往往只使用了语言功能的一个子集,而这个子集并不一定是以面向对象为核心。相反,真正掌握这门语言的强大之处,在于勇于尝试各种编程范式,充分利用 C++ 的灵活性与表现力,根据当前任务的实际需求选择最合适的实现方式。

在下一章中,我们将进一步深入,探讨一个常常被忽视的问题:main() 函数真的就是你程序的起点吗?答案可能会让你感到意外。