在本章中,我们看到:可以用多种范式在 C++ 中进行编程。我们简要探讨了几种不同的编程风格,包括函数式编程、元编程、确保编译时验证的类型系统,以及极端多态性。这些方法,连同标准的面向对象编程和结构化编程,在构建类库或特定程序时都适用于不同的上下文场景。对于那些渴望尽可能深入了解自己技艺的好奇程序员来说,每一种范式都能带来独特的收获。它们各自都有权衡取舍,并在软件开发世界中有着不同的实现方式。

我们已经展示过,C++ 程序员可能只使用了语言的一个子集,而且这个子集并不一定是面向对象的。相反,最好的做法是去尝试所有这些范式,充分利用 C++ 强大而灵活的语言特性,根据当前任务的需要选择最合适的方式。

在下一章中,我们将看到:main() 函数实际上可能并不是我们应用程序的真正入口点。