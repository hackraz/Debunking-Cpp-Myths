
如果你像我一样,经常在不同的组织、团队和专业技术会议之间穿梭往来,你会很快注意到两件事:C++ 开发者的兴趣与其他开发者截然不同,而且 C++ 社区实际上是由众多小型、专业化的开发者群体组成的。这种现象在其他编程语言社区中并不常见。

\begin{itemize}
\item 
谈论 Java 时,讨论往往集中在 Spring 框架和 REST API 或 Android 工具包

\item 
C\# 主要围绕微软的技术栈展开,高度标准化

\item 
JavaScript 几乎总是离不开 React 或富客户端网页开发
\end{itemize}

然而,当你将一百位来自不同背景的 C++ 开发者聚集在一个房间里,你会发现他们之间的巨大差异。这些差异反映了他们在各自领域中的独特需求和挑战:

\begin{itemize}
\item 
嵌入式 C++ 开发者 关注的是对所有资源的严格控制。对于销量以百万计的设备而言,哪怕增加 1MB 的内存都会迅速推高成本。

\item 
高频交易领域的开发者 对避免 CPU 缓存未命中有着深刻的理解,并且知道如何通过优化自动化交易算法来“节省”几皮秒(picosecond)的时间 --- 因为最微小的时间单位可能意味着数百万欧元的差距。

\item 
工程软件开发者 相对宽松一些,但仍需确保复杂渲染模型中更改的正确性。

\item 
铁路、汽车或工厂自动化系统的开发者 则主要关注系统的弹性和鲁棒性。
\end{itemize}

虽然这幅图景远非完整,但它已经足以展示 C++ 开发者之间的巨大多样性,这是其他任何语言的开发者群体所无法比拟的。我们甚至可以说,C++ 是目前仍是广泛使用的一种通用编程语言。其他主流语言大多仅用于特定类型的程序:Java 用于企业后端服务和 Android 开发,C\# 用于 Web 和 Windows 应用及服务,JavaScript 用于富客户端网页和无服务器后端,Python 用于脚本、数据科学和 DevOps。而 C++ 却广泛应用于嵌入式软件、工厂系统、金融交易、仿真、工程工具、操作系统等多个领域。

那句老话“形式追随功能”原本是用来描述人类建造的一切事物的设计理念,包括编程语言。这句话同样适用于 C++。项目和开发者类型的巨大多样性被融入到了语言之中,再加上 Bjarne Stroustrup 希望它尽可能强大的愿景,使得 C++ 成为了一个多功能的语言集合体。

C++ 并不是一门单一的语言;每一位开发者都在使用 C++ 的一个子集,而这往往与其在同一组织工作的同事所使用的部分大相径庭。

C++ 起源于面向对象编程兴起的时代,最初只是“带有对象的 C”。但与此同时,C++ 与 C 保持了向后兼容,所以仍然可以在 C++ 中编写结构化编程风格的代码。后来模板变得不可或缺,再后来 lambda 表达式也变得实用。虽然 C++ 一直以来都是多种编程范式的集合体,但在今天,这种特性更加明显。

为了证明这一点,让我们来看看几种你可以在 C++ 中使用的编程范式,首先从函数式编程开始。