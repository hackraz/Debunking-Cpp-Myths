
亲爱的读者,若您还记得第九章那个有趣的章节《零的定义》,那就再好不过了——因为我们即将再次探讨这个极具影响力的数字。若您一时想不起来,也无需担心:既然您已购得完整版书籍,大可随时翻回第九章重温(就当是温故知新吧)。

现在请看以下程序:

\begin{cpp}
#include <iostream>
#include <typeinfo>
#include <string>

template<typename T> std::string typeof(T t) {
  std::string res = typeid(t).name();
  return res;
}

int main() {
  auto a1 = 0;
  auto a2(0);
  auto a3 {0};
  auto a4 = {0};
  std::cout << typeof(a1) << std::endl
            << typeof(a2) << std::endl
            << typeof(a3) << std::endl
            << typeof(a4) << std::endl;
}
\end{cpp}

这段程序依然不算复杂,它只是巧妙地运用了auto关键字,并通过前文章节介绍的各种机制将变量初始化为0。若您还不清楚auto的用途,这里简要说明:C++11中的auto关键字是从C语言"征用"而来的,其新使命是实现自动类型推断——让编译器能根据初始化表达式推导变量类型。这种设计通过消除显式类型声明的需求,不仅简化了代码,更让迭代器或模板类型等复杂冗长的类型处理变得简洁。

回到代码分析,经过仔细推敲我们可以得出以下结论:

\begin{itemize}
\item 
\verb|auto a1 = 0;|:这是最基础的拷贝初始化,由于0是整型字面量,a1被推导为int类型

\item 
\verb|auto a2(0);|:同样简单,通过直接初始化整型字面量0,a2也被推导为int

\item 
\verb|auto a3 {0};|:采用列表初始化{0}时,a3仍被推导为int(注意:此行为在C++17前后有变化)

\item 
\verb|auto a4 = {0};|:这里有个特殊规则:当auto遇见带等号的大括号初始化时,会优先推导为\verb|std::initializer_list<int>|,这是C++11为统一初始化引入的特例
\end{itemize}

使用MSVC编译时,程序输出如下:

\begin{shell}
int
int
int
class std::initializer_list<int>
\end{shell}

若使用较新版本的GCC编译,输出会相对简洁(但核心逻辑不变):

\begin{shell}
i 
i 
i
St16initializer_listIiE
\end{shell}

然而这里有个陷阱——如果用5.0版本之前的GCC编译,你会看到如下"惊喜"输出:

\begin{shell}
i 
i
St16initializer_listIiE
St16initializer_listIiE
\end{shell}

这可谓"向后兼容性"带来的意外"惊喜"。不过真正的救星是Clang(3.7+版本),其编译时会给出极具参考价值的提示信息:

\begin{shell}
<source>:19:13: warning: direct list initialization of a variable
with a deduced type will change meaning in a future version of Clang;
insert an '=' to avoid a change in behavior [-Wfuture-compat]
  auto a3 {0};
\end{shell}

由此可见,在C++17标准诞生前后,auto与\{x\}组合的语义(在这个特定场景下)发生了微妙变化。不过值得庆幸的是,早期编译器开发者早已预见这种特殊情况,给出了非常直接的警告——这"向后兼容"做得可真够别扭的,不是吗?

有了这些铺垫,以下代码无法编译的现象就不足为奇了(假设仍沿用之前的简短程序框架):

\begin{cpp}
std::cout << typeof( {0} );
\end{cpp}

为什么应该这样做呢?考虑到前面的语法的所有混乱和混乱,{0}会被推断为哪种类型?它被推导出一个 int 类型吗?或者也许是 initializer\_list 类型? 它会是空指针 (nullptr) 吗或者可以从数字构建的对象,如下所示:

\begin{cpp}
struct D { D(int i) {} };
void fun(D d) { }
fun({0});
\end{cpp}

现在是不是觉得这个玩笑没那么有趣了?








