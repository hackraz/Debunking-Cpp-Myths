在本章中我们已经看到,C++拥有覆盖各种需求的丰富库和框架。但与其他技术相比,获取这些资源的过程并不简单:它们没有集中存放的平台导致难以发现,还可能带来编译器兼容性或陈旧代码风格等额外问题——这正是我们在本章学到的关键认知。

与其他技术类似,C++库同样存在漏洞风险,容易遭受供应链攻击。防范措施包括持续跟踪漏洞披露动态、下载时验证二进制文件真实性。正如本章所述,额外的代码审计和漏洞扫描始终有益。因此大型组织在安全方面更具优势——他们拥有专门团队处理这些问题,代价则是牺牲了灵活性。

那么,C++是否存在支持现代编程的库?答案当然是肯定的。只是相比其他主流技术,它们更难以发现且兼容性更差。

下一章我们将探讨:C++是否具备向后兼容性——既包括语言自身的迭代,也涉及更广泛的生态兼容。