
让我们暂时跳出C++世界,化身墙头苍蝇观察其他技术栈开发者的工作流程。从项目创建到团队扩容,典型流程如下:开发者通常会启动微软系(如Visual Studio .NET/VS Code)或JetBrains系(如IntelliJ IDEA/PyCharm/Rider)IDE创建项目。少数异类(比如我)会选择命令行+neovim,更古怪的则用Emacs——当然这是玩笑,毕竟《真正程序员》xkcd漫画(\url{https://xkcd.com/378/})告诉我们,硬核程序员都靠大气电场直接操控比特。

新建项目时,IDE会推荐集成库。虽然初始项目功能有限,但创建过程会关联Git代码库,本地提交后即可推送到共享仓库。团队成员只需克隆仓库,IDE便会自动处理项目配置中的依赖项。

此时项目通常已包含日志和单元测试库。Python标准库自带logging,Java社区偏好Log4J,.NET则采用Microsoft.Extensions.Logging或Log4Net。单元测试方面,Python虽提供标准库但开发者更爱pytest;Java需搭配JUnit/TestNG和Mockito/JMock;.NET虽有内置框架,但技术负责人往往选择NUnit/xUnit+Moq。

那么C++生态如何?日志库早已随C++共同成熟,功能趋同但API略有差异,以致选择困难——许多项目直接使用Boost.Log。GitHub上24k星的spdlog(仅支持C++11)便是例证。单元测试库更是百花齐放:Google的GTest/GMock、标准xUnit结构的CppTest、本书偏好的无依赖单头文件库doctest(\url{(https://github.com/doctest/doctest}),以及适合嵌入式开发的CppUTest(\url{http://cpputest.github.io})。Mocking框架推荐易集成的FakeIt(\url{https://github.com/eranpeer/FakeIt})。

这些库通常通过包管理器管理,依赖清单以文本/标记/脚本文件形式存入代码库。企业环境因安全考虑会限制依赖添加:或限定内部源白名单,或需逐包审批,或设置权限门槛。

无论如何,当新成员加入时,只需克隆中央代码库并执行安装命令——通常只需在IDE中加载项目,剩下的交给自动化工具,一切便能正常运行。至此,我们的"墙头苍蝇"观察记告一段落。

让我们深入探究包管理器背后的运作机制。作为常年在Ubuntu Linux上使用neovim+命令行开发的开发者,我对各技术栈的流程略知一二。以Python为例,推荐使用虚拟环境避免污染系统库。工具链pipenv将标准库提供的pip包管理器与venv虚拟环境相结合,实现简易配置。典型命令行操作如下:

\begin{shell}
pipenv init
pipenv install [library name]
\end{shell}

在新环境中,只需运行以下命令即可安装所有依赖项:

\begin{shell}
pipenv install
\end{shell}

Java和.NET的工作流程类似,只是无需虚拟环境。它们分别使用开源包管理器:Java用Maven或Gradle,.NET用NuGet。这三种技术都有统一的库托管平台:Python的PyPI(\url{https://pypi.org/})、Java的Maven中央仓库(\url{https://mvnrepository.com/repos/central})和.NET的NuGet站点(\url{https://www.nuget.org/})。如前所述,大公司会更严格控制库的使用,往往自建仓库(如Java的Artifactory \url{https://jfrog.com/artifactory/})。

如今开发者可以轻松搜索、更新和安装库,这些已成为各技术栈的标准实践。相比2000年代需要手动下载二进制文件或自行编译的困境,现代C++通过Conan和vcpkg正在缩小差距。大公司的程序员可能已习惯类似Java/Python/.NET的体验——使用内部审核通过的Conan/vcpkg仓库。虽然新增库到白名单流程繁琐,但可以理解。

若无完善的基础设施,体验就大打折扣:库分散各处,工具也不够稳定。笔者亲历Conan在简单项目上报错却无从解决的窘境。虽然讨厌Maven即使简单配置也要下载大量包,但它至少稳定可靠——这正是包管理器的核心价值。因此不得不承认:尽管C++包管理在追赶,但尚未成熟。

当然,大公司的开发者可能无此困扰。假设包管理器运作良好,接下来根据项目需求,我们还需更多功能库。让我们看看主要类别。




