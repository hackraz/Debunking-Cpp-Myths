在思考本章标题所提出的问题时,我们首先面临选择困境。库的选择完全取决于项目具体需求 --- 虽然日志记录、单元测试等功能属于通用需求,但除此之外该如何取舍?毕竟,将 C++ 的 Web 开发与 Java 的 Web 开发对比,或将 C++ 系统编程与 Java 系统编程比较,都显失公允。事实上,C++ 很少用于 Web 开发,Java 也罕用于系统编程。

C++长期占据着独特生态位:尽管受到Java、C\#、Rust和Python的蚕食,它仍在游戏开发、固件、高频交易、工程应用、汽车电子、系统编程等领域坚守阵地。其他语言难以渗透这些领域,根源正在于C++无与伦比的灵活性、性能和控制力。

另一重挑战在于库的数量级差异。作为一门悠久的语言,C++ 拥有像 Boost 这样在 Java/C\# 世界(标准库除外)无可匹敌、在 Python 领域更无对手的巨型库。或许只有 JavaScript 的 React 及其生态可堪比拟。就库的数量而言,C++ 显然占据优势。

这些观察引出一个关键特征:库的现代化程度。我们对现代编程语言及其生态有何期待?从这个角度看,C++ 处于什么位置?让我们深入探讨这些问题。