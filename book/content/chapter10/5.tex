
但凡关注技术安全的人都清楚,软件始终存在安全隐患。随着软件应用渗透日常生活的方方面面,这一问题正日趋严峻。

提升安全性需要双管齐下:网络安全专家负责发现漏洞并构建防护工具,而软件开发人员则需在发布前排查安全问题并管控风险。我们深知没有绝对安全的软件,但也明白现状可以大幅改善。

其中关键环节在于管理第三方库的潜在漏洞,这些漏洞主要来自两种情况:无意引入的缺陷,或恶意攻击者故意注入的后门。

客观而言,所有技术栈都面临此风险。许多知名C++库都经过大型企业的安全审查——如果您就职于大公司,自有专业团队处理这些隐患。但并非所有开发都在大公司进行,也并非所有库都受到同等重视,正如xz后门事件所警示的那样。让我们重点讨论第二种情况,恶意攻击者主要通过以下方式注入漏洞:

\begin{itemize}
\item 
代码贡献渗透:通过向开源项目提交"有用"的代码贡献,夹带漏洞。典型案例是2021年发现的有意破坏Node.js事件,攻击者试图提交看似合理的ES模块兼容性修复,实则包含内存破坏漏洞。

\item 
项目分叉污染:克隆知名开源项目后,在新增功能中植入恶意代码。例如2018年的event-stream事件,攻击者接管维护权后,在加密功能中注入窃取比特币钱包的代码。

\item 
维护者身份劫持:通过社会工程学手段成为项目维护者后实施入侵。如震惊业界的xz后门事件(详见笔者深度分析:\url{https://mozaicworks.com/blog/xz-backdoorand-other-news}),攻击者经过两年潜伏才触发后门。

\item 
二进制文件替换:在非官方渠道发布带漏洞的编译版本,或劫持发布流程。PyPI就曾多次出现仿冒知名库的恶意包,名称仅差一个字符。

\item 
下载链路劫持:通过DNS污染或本地hosts文件篡改,将下载请求重定向到恶意服务器。2018年GitHub曾遭遇中间人攻击,导致部分用户下载被注入的代码。
\end{itemize}

前述列表中的所有问题都相当严重。大型企业的安全部门和IT运维团队通常会关注这些风险,但对于小型企业而言,可能需要额外提高警惕。目前已知的解决方案是通过数字签名或哈希值验证所有二进制文件。

虽然编程语言包管理器和Linux系统会自动执行这种验证,但从GitHub手动下载二进制文件时,则需人工验证签名——这要求开发者同时提供配套的库文件签名。

第一种情况则更为复杂:如何判断某个库是否存在漏洞?对于开源代码,普遍观点是"众人审查"能发现所有问题。但这种观点高度依赖于项目贡献者数量及其专业水平。

提到的xz后门事件令人不寒而栗,尤其是问题发现者——开发者Andres Freund仅仅因为在微基准测试中注意到sshd进程CPU占用异常而起了疑心(\url{https://mastodon.social/@AndresFreundTec/112180406142695845})。

这一事件短暂暴露了开源库维护者过度劳累的问题,但很快又重归沉寂。

即便假设多数开源库没有被获得维护者权限的恶意分子攻击,漏洞依然可能潜伏——C++在这方面尤为突出,因其自身的安全特性本就存在挑战。小型团队必须持续关注所用库的已报告漏洞,或者购买能自动监控的安全工具。

即便一切运行正常,最佳实践仍是维护应用程序的依赖库清单,以便运维人员定期检查所有使用库的漏洞。该领域的推荐方案是创建所谓"软件物料清单(SBOM)",其中包含所有库及其依赖项的详细记录。现有工具(如Grype \url{https://github.com/anchore/grype} 及其配套工具Syft \url{https://github.com/anchore/syft})可自动生成SBOM并基于清单扫描漏洞,不过多数工具仅支持Docker容器环境。

由此,我们得出本章的结论。







