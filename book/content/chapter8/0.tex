\begin{flushright}
\textit{--- 这已经是性能的底线}
\end{flushright}

在追求极致效率的C++开发领域,优化代码以榨取最后一丝性能始终是令人着迷的挑战。这条探索之路常常将开发者引向计算的本质——当C++邂逅汇编语言,每个CPU时钟周期都弥足珍贵。

三十年前的九十年代,程序员们常常需要手工打磨每个字节的机器码,甚至潜入汇编语言(乃至更低层)的深水区来达成预期性能。这些优化技术的先驱者虽然以今日标准看来方法原始,却为理解C++与汇编的威力与局限奠定了基础。

本次探索聚焦于一个看似简单的任务——点亮屏幕像素的优化,通过对比三十年前手工优化的汇编例程与现代Clang/GCC/MSVC等先进编译器的输出,我们将见证编译器技术的演进历程。随着分析深入,您将看到人类直觉与机器优化之间的平衡如何变迁,从而洞察我们编写的代码与最终运行程序的机器之间不断演进的关系。(注:本章聚焦Intel x86处理器架构特性,ARM架构将留待其他著作探讨)

本章您将学到:

\begin{itemize}
\item 
如何运用汇编代码加速关键例程

\item 
为何有时应避免使用汇编而信任编译器优化器的决策
\end{itemize}


