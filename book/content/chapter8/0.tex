\begin{flushright}
\textit{--- 不可复降于此}
\end{flushright}

在追求极致效率的 C++ 开发领域,优化代码以榨取每一丝性能始终是一项令人着迷的挑战。这条探索之路往往将开发者引向计算的本质 --- 当高级语言与底层硬件相遇,每一个 CPU 时钟周期都变得弥足珍贵。

回到三十年前的九十年代,开发者们常常需要手工打磨每一个字节的机器码,甚至深入汇编语言(乃至更底层)的“深水区”,只为实现那微乎其微却至关重要的性能提升。尽管这些早期优化手段在今天看来或许显得原始,但它们为理解 C++ 与汇编语言之间的关系、以及性能优化的本质奠定了坚实基础。

本章将聚焦于一个看似简单的任务 --- 点亮屏幕像素的优化。我们将通过对比三十年前手工优化的汇编例程与现代 Clang、GCC、MSVC 等先进编译器的输出结果,揭示编译器技术在过去几十年中的演进历程。随着分析的深入,你将看到人类直觉与机器优化之间不断变化的平衡点,从而洞察我们所写的代码与最终运行它的机器之间日益复杂的关系。(注:本章聚焦Intel x86处理器架构特性,ARM架构将留待其他著作探讨)

通过本章,你将掌握以下关键知识:

\begin{itemize}
\item 
如何通过嵌入式汇编或内联汇编加速关键性能路径上的例程。

\item 
在哪些场景下应避免手动编写汇编代码,而应信任现代编译器强大的优化能力。
\end{itemize}


