闪光的不都是金子,看似炫酷复杂的代码也未必优质。那些花哨精巧的代码往往掩盖了优秀编程实践的本质——好的、稳定的代码通常以直白和可预测性见长,而非华丽技巧。相比趣味性强的代码结构,这类代码或许显得平淡无奇,但正是这种朴素确保了其健壮性和易理解性。

当工作需要时,请务必编写"无聊"的简单代码——毕竟半年后重读时会轻松许多;但在个人趣味项目中,不妨塞进一两只小熊(除非你打算日后还要阅读这些代码)。

下一章中,Alex将发起一场现代C++库的正确使用指南,彻底粉碎"C++库还停留在石器时代"的谬见。