闪光的不都是金子,看似炫酷复杂的代码也未必就是优质代码。那些花哨精巧的实现,往往掩盖了优秀编程实践的本质。

真正的好代码,通常是直白的、稳定的、可预测的 --- 而不是堆砌技巧、追求炫技。相比趣味十足、充满“创意”的代码结构,这类代码或许显得平淡无奇,但正是这种朴素,确保了它的健壮性与可维护性。

当你在工作中编写代码时,请务必选择那种半年后你还能轻松看懂的“无聊”写法。毕竟,六个月后的你,可能已经不是现在的你 --- 而那段“聪明”的代码,可能会让你抓狂不已。

但在个人项目或趣味编程中?当然可以放飞自我!塞进一两只小熊也没关系 --- 只要你不打算日后还得读懂它就行。

下一章中,Alex 将带领我们走进现代 C++ 库的世界,带来一份关于C++ 标准库与现代第三方库正确使用方式的实用指南。我们将彻底粉碎那个流传已久的谬见:“C++ 的库还停留在石器时代”。

准备好迎接一场高效、优雅、现代化的 C++ 编程之旅了吗?