
当今三大主流 C++ 编译器中的两个 --- GCC 和 Clang --- 是以开源方式开发和维护的。从理论上讲,任何人都可以为这些编译器贡献代码,提出新特性,甚至修复问题。然而在实践中,真正推动其发展的,往往是少数具备专业知识、持续投入且通常受到大型企业支持的核心开发者。

截至 2024 年,GCC、Clang 以及我们在上一节中提到的 Microsoft Visual C++(MSVC),是当前最符合 C++ 标准的编译器,三者之间难以简单地分出高下。尽管它们都致力于实现标准规范,但这并不意味着它们没有各自独特的扩展功能。事实上,许多开发者认为,合理使用这些扩展反而能提升代码的表现力与效率。

以 GCC 的“计算 goto”功能为例(Clang 同样支持这一特性),它便是一个极具争议但又颇具实用价值的非标准扩展。我们曾在学校里学到:goto 是邪恶的,应该永远避免使用。如果你在学校没听过这句话,也请不要从本书中学到它 --- 因为这次,“它其实并不完全正确”。

与其说 goto 是“邪恶”的,不如说它是一种强大而容易滥用的语言特性。那么,当我们谈到“计算 goto”时,我们是否可以说它是“经过计算的邪恶”呢?下面这段代码究竟属于“纯粹的邪恶”,还是“经过计算的邪恶”?让我们来看一个示例:

\begin{cpp}
int main() {
  std::vector<void*> labels = { &&start, &&state1, &&state2, &&end };
  int state = 0;
  goto *labels[state];
  start:
    std::cout << "In start state" << std::endl;
    state = 1;
    goto *labels[state];
  state1:
    std::cout << "In state 1" << std::endl;
    state = 2;
    goto *labels[state];
  state2:
    std::cout << "In state 2" << std::endl;
    state = 3;
    goto *labels[state];
  end:
    std::cout << "In end state" << std::endl;
    return 0;
}
\end{cpp}

第一行代码并没有问题。真正的挑战从那之后才开始。这个非常实用但非标准的功能 --- 计算 goto --- 允许通过指针跳转到特定的标签位置。这些标签地址可以在运行时动态选择,并用于实现高效的解释器、虚拟机或状态机等结构。标签指针通常通过一个指针数组初始化,而这些指针又指向各自标签的实际地址。

由于它操作的是指针,因此我们甚至可以使用“令人望而生畏”的指针运算来对地址进行一些动态调整。这在某些底层性能敏感的场景中极具吸引力。

然而,与任何强大的工具一样,它也可能带来风险。与标准 goto 不同的是,计算 goto 并不会考虑对象生命周期的管理。如果跳过了某个具有自动存储期的对象的构造点,或者跳出了它的作用域,那么该对象的析构函数将不会被调用。这可能导致资源泄漏、未定义行为,甚至是难以追踪的 bug。

所以,请把它当作一种“受控的危险”:只有在你完全理解后果的情况下才应使用它。

另一个来自 GCC 的有趣扩展(同样也被 Clang 支持,真是令人惊喜),也展示了一种对标准 C++ 语法的实用偏离。它使得以下这段代码能够在上述两个编译器上顺利编译:

\begin{cpp}
int y = ({ int x = 10; x + 5; });
\end{cpp}

很整洁,不是吗?这个特性被称为 “表达式中的语句和声明”,它带来了你能想象到的各种好处:内部声明的对象可以被很好地封装,如果使用得当,甚至可以让宏的使用变得更加安全。

可惜的是,它并不是标准 C++ 的一部分。

Clang 作为编译器世界中的“新秀”\footnote{好吧,如果我们能称一个已有 15 年历史的编译器为“新”的话 --- 与 GCC 相比,它的诞生时间确实要晚很多,毕竟 GCC 早在 1987 年就已问世。但无论如何,Clang 仍然是一个相对年轻却技术精湛的参与者},在支持这类创新特性方面走得更远。

下面这段代码只能在 Clang 下成功编译,并且得益于其特有的编译器选项和新兴语言特性的支持:

\begin{cpp}
#include <iostream>
int main() {
  int (^square)(int) = ^(int num) { return num * num; };
  int y = square(12);
  std::cout << y << std::endl;
}
\end{cpp}

这一特性被称为 Clang 中的 Blocks。要正确使用它,你需要安装 BlocksRuntime\footnote{\url{https://github.com/mackyle/blocksruntime}} 库,并在编译时为 Clang 指定一个特殊的标志:-fblocks。完成这些配置之后,你就可以成功编译前面的代码了。

从行为上看,Blocks 与 C++11 引入的 lambda 表达式非常相似。考虑到 Blocks 是在 2008 年由 Clang 首先提出并实现的,我们可以说它是标准 C++ lambda 的“前身”或“灵感来源”。

如果你对它们之间的关系感兴趣,下面是用标准 C++ 编写的等效 lambda 示例:

\begin{cpp}
auto square = [](int num) ->int { return num * num; };
\end{cpp}

所以,如果你刚才看到类似 Blocks 的语法时感到有些困惑,那可能只是你对 C++ 新特性的记忆需要一次小小的“刷新”。

顺便提一句,在任何情况下,下面这种写法都不是标准 C++:

\begin{cpp}
int array[n] = {0};
\end{cpp}

这是一个可变长度数组,是 C 语言中的一项功能。由于存在诸多安全和可移植性问题,C++ 标准并未将其纳入其中。

尽管如此,GCC 编译器仍然接受上述写法,而 Clang 则会报错:

\begin{shell}
error: variable-sized object may not be initialized
   5 | int array[n] = {0};
\end{shell}

根据错误提示,修复方法其实很简单。可以将原始代码改写为更符合现代 C++ 风格的形式:

\begin{cpp}
auto generate(int n) -> std::vector<int>{
  int array[n];
  for(int i=0; i<n; i++) array[i] = i;
  return std::vector<int>{array, array + n};
}
\end{cpp}

这样修改后,即使是 Clang(以及其他主流编译器,如 ICC)也能顺利通过编译,无论这段代码是否完全符合标准……或者说,更准确地说 --- 体现了标准之外的现实世界。

\mySubsubsection{2.4.1}{敬属性}

令人惊讶的是,GCC、Clang 和 Microsoft Visual C++ 在 C++ 语言的一个非常具体的扩展上达成了共识:我们需要一种机制,能够将元数据附加到某些语言结构(如类型、函数、变量等)之上。这些元数据随后可以被编译器或其他工具使用,用于生成优化代码、执行静态检查或提供额外功能。

在现代 C++(即 C++11)引入标准化的双方括号语法 [[attribute]] 来指定属性之前,各主流编译器都各自实现了自己的方式来声明这些关键属性:

\begin{itemize}
\item 
GCC 和 Clang 使用 \_\_attribute\_\_((attribute-name))

\item 
Microsoft Visual C++ 使用 \_\_declspec(attribute-name)
\end{itemize}

随着 C++11 的发布,标准化委员会认识到这些属性的重要性和实用性,并将其中最常用的一些提升为语言标准的一部分,例如 [[noreturn]]。后续的标准更新也不断引入新的属性,如 [[fallthrough]]、[[nodiscard]] 等。

然而,许多其他有用的属性仍然局限于它们最初诞生的编译器。下面的代码段展示了一些典型的非标准但广泛使用的属性示例:

\begin{cpp}
void old_function() __attribute__((deprecated));
void fatal_error() __attribute__((noreturn));
int pure_function(int x) __attribute__((pure));
int x __attribute__((aligned(16)));
void old_function() {
  std::cout << "This function is deprecated.";
}
void fatal_error() {
  std::cerr << "This function does not return.";
  exit(1);
}
int pure_function(int x) {
  return x * x;
}
\end{cpp}

上述代码中展示了 GCC 和 Clang 支持的一些常见属性及其用途:

\begin{itemize}
\item 
\_\_attribute\_\_((deprecated))  标记old\_function为已弃用,提醒开发者避免使用。

\item 
\_\_attribute\_\_((noreturn)) 告知编译器 fatal\_error 函数不会返回,有助于优化和警告抑制。

\item 
\_\_attribute\_\_((pure))  表示 pure\_function 是一个纯函数,除了返回值外没有副作用,可用于进一步优化。

\item 
\_\_attribute\_\_((aligned(16))) 将变量 x 对齐到 16 字节边界,常用于性能敏感或硬件交互场景。
\end{itemize}

这些编译器 \footnote{\url{https://gcc.gnu.org/onlinedocs/gcc/Function-Attributes.html}} 提供的属性列表非常庞大 \footnote{\url{https://clang.llvm.org/docs/AttributeReference.html}}。如果正在某个特定平台上进行开发,并且主要关注点不是代码的可移植性、平台独立性或严格的标准合规性,强烈建议对相关文档进行查阅。

通过正确使用编译器提供的扩展属性,可以充分利用其强大的功能,在性能、安全性和代码表达力方面获得显著提升。


