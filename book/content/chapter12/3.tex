
要理解Rust的核心特性,首推其官网(\url{https://www.rust-lang.org/}),该站对其核心优势的提炼极为精当:

\begin{itemize}
\item 
运行高效且内存节约

\item 
零运行时开销

\item 
无垃圾回收机制

\item 
卓越的跨语言互操作性

\item 
通过丰富的类型系统和所有权模型实现内存安全和线程安全

\item 
出色的文档

\item 
友好的编译器,提供有用的错误消息

\item 
集成的包管理器和构建工具

\item 
自动格式化程序

\item 
智能多编辑器支持,具有自动完成和类型检查
\end{itemize}

仅从这些描述中,我们就能看出Rust与C++的诸多相似之处,以及它对C++现状的改进。相似点主要体现在控制力层面:原生编译、无垃圾回收、高性能和内存效率——这些同样是C++标榜的特性。差异则指向本书反复讨论的痛点:标准化的包管理、统一工具链以及友好的编译器。最后这点尤其让饱受冗长错误信息折磨的C++程序员热泪盈眶——记得2000年代使用Visual C++时,我曾遇到过编译器直接提示"错误信息过长无法显示"。虽然现代C++已有所改善,但模板相关的错误排查仍是场噩梦。

不过,让我们超越官网首页的宣传语。接下来我将重点介绍几个相较于C++特别实用且有趣的功能特性:

\mySubsubsection{12.3.1}{项目模板系统和包管理机制}

作为命令行和Neovim代码编辑器的重度用户,我尤其钟爱能直接从命令行创建项目的技术。Rust自带的cargo工具可以完成项目创建、构建、运行、打包和发布全流程工作:

\begin{itemize}
\item 
创建项目:cargo new 项目名

\item 
运行项目:cargo run

\item 
检查编译错误:cargo check

\item 
编译项目:cargo build

\item 
打包项目:如您所料——cargo package

\item 
发布项目:(鼓点声)...cargo publish
\end{itemize}

通过cargo我们不仅能创建库和可执行文件,更能使用cargo generate工具(参见、\url{https://cargo-generate.github.io/cargo-generate/})基于项目模板快速初始化。

我知道这对大多数C++开发者可能不足为奇——毕竟你们很少创建新项目。在教授C++程序员单元测试或测试驱动开发时,这点常让我惊讶:我们需要共同费劲地配置测试项目与生产项目的引用关系,而这本应是理所当然的配置。请相信,这种便捷性不仅体现在项目初期,对小型实验、个人练习代码库乃至缩短编译时间都大有裨益。在SSD硬盘显著提升编译速度前,我经常使用C++的一个技巧:将正在修改的少数文件组成新编译单元,其余部分作为二进制引用——这算是C++应对缓慢编译的土法方案。

关于新建项目就说到这里。现在让我们写点代码...或者说,尝试修改些变量...(突然想起Rust的不可变性)

\mySubsubsection{12.3.2}{不可变性}

Rust 默认采用不可变设计,正如官方文档所述:"一旦值被绑定到变量名,就无法更改该值"。让我们通过一个简单示例演示:先为变量分配字符串值并显示,然后尝试修改它:

\begin{rust}
fn main() {
  let the_message = "Hello, world!";
  println!("{the_message}");
  the_message = "A new hello!";
  println!("{the_message}");
}
\end{rust}

尝试编译此程序会导致编译错误。

\verb|cannot assign twice to immutable variable the_message| 

贴心的错误信息还包含提示:

\verb|For more information about this error, try 'rustc -{}-explain E0384'|

错误解释不仅包含示例,还明确说明如何声明可变变量:

\textit{"Rust变量默认不可变。要修复此错误,需在声明变量时于let后添加mut关键字"。}

以下是修改后可正常编译的代码示例:

\begin{rust}
let mut the_message = "Hello, world!";
println!("{the_message}");
the_message = "A new hello!";
println!("{the_message}");
\end{rust}

由此可见,可变变量必须显式声明为 mut,这体现了Rust默认不可变的设计哲学。正如前几章讨论的,这种设计能有效解决并行并发、自动化测试和代码简洁性等诸多问题。

\mySubsubsection{12.3.3}{复合类型的简洁语法}

Rust 借鉴了 Python 和 Ruby 等语言的语法特性来处理数组和元组,其声明方式如下所示:

\begin{rust}
let months = ["January", "February", "March", "April", "May", "June", "July", "August", "September", "October", "November", "December"];
println!("{:?}", months);

let (one, two) = (1, 1+1);
println!("{one} and {two}");
\end{rust}

这看似微不足道,却能显著简化代码。值得一提的是,C++自C++11起通过花括号列表初始化(list initializer)引入了类似语法,并在后续版本中持续优化:

\begin{rust}
std::vector<string> months = {"January", "February", "March", "April", "May", "June", "July", "August", "September", "October", "November", "December"};
\end{rust}

我期待这方面能有更多改进,但考虑到C++语法已然十分复杂,对此并不抱太高期望。

\mySubsubsection{12.3.4}{可省略的return关键字}

Rust函数允许直接返回最后一个表达式的值。以下示例演示了如何使用该特性实现数字自增:

\begin{rust}
fn main() {
  let two = increment(1);
  println!("{two}");
}

fn increment(x:i32) -> i32{
  x+1
}
\end{rust}

虽然在前述简单函数中我通常不会使用这种写法,但省略return关键字确实能简化闭包编写——这正是我们接下来要探讨的内容。

\mySubsubsection{12.3.5}{闭包}


让我们对向量中的所有元素进行自增操作:

\begin{rust}
fn increment_all() -> Vec<i32>{
  let values : Vec<i32> = vec![1, 2, 3];
  return values.iter().map(|x| x+1).collect();
}
\end{rust}

与函数式编程的常规做法类似(如同C++中的ranges库),我们需要:

\begin{itemize}
\item 
获取迭代器

\item 
调用map函数(相当于C++中的transform算法)并传入闭包

\item 
调用collect获取结果集
\end{itemize}

闭包语法之所以如此简洁,正是得益于可选的return语句设计。

\mySubsubsection{12.3.6}{标准库中的单元测试}

单元测试是软件开发中至关重要的实践,但令人惊讶的是,仅有少数语言在标准库中提供原生支持。Rust默认集成了这一功能,且使用极其简便。让我们为increment\_all函数添加单元测试以验证其行为符合预期:

\begin{rust}
#[cfg(test)]
mod tests {
  use super::*;

  #[test]
  fn it_works() {
    assert_eq!(vec![2, 3, 4], increment_all());
  }
}
\end{rust}

额外值得一提的是,我特别欣赏Rust允许在同一个编译单元(Rust中称为crate)内同时编写生产代码和单元测试的便捷性。如果您仅将单元测试视为义务性工作,这可能无足轻重;但对我这样常通过单元测试进行技术探索和设计验证的开发者而言,此特性实属福音。

\mySubsubsection{12.3.7}{特质}

Rust(以及Go)与其他主流语言的一大区别在于:它不支持继承机制,而是推崇组合模式。为实现多态行为,Rust提供了特质(trait)这一特性。

Rust的特质类似于面向对象语言中的接口,它们定义了一组需要由派生类型实现的方法。但Rust特质有个独特功能:您可以为非自身定义的类型添加特质。这与C\#的扩展方法类似,但又不尽相同。

Rust官方文档通过两个结构体示例演示了特质的使用:一个表示推文(tweet),另一个表示新闻文章(news article)。为两者添加Summary特质后,就能生成对应的内容摘要。如下例所示,特质实现既独立于结构体实现,也独立于特质定义本身,这种设计提供了极大的灵活性。

首先让我们看看这两个结构体的定义。新闻文章结构体NewsArticle包含以下字段:

\begin{rust}
pub struct NewsArticle {
  pub headline: String,
  pub location: String,
  pub author: String,
  pub content: String,
}
\end{rust}

然后,Tweet 结构包含其自己的字段:

\begin{rust}
pub struct Tweet {
  pub username: String,
  pub content: String,
  pub reply: bool,
  pub retweet: bool,
}
\end{rust}

另外,我们使用返回字符串的单个方法 summarize 来定义 Summary trait:

\begin{rust}
pub trait Summary {
  fn summarize(&self) -> String;
}
\end{rust}

现在,我们来为 Tweet 结构实施 Summary 特征。这是通过指定此 trait 的实现应用于结构来成的,如下所示:

\begin{rust}
impl Summary for Tweet {
  fn summarize(&self) -> String {
    format!("{}: {}", self.username, self.content)
  }
}
\end{rust}

测试完美运行:

\begin{rust}
#[test]
fn summarize_tweet() {
  let tweet = Tweet {
    username: String::from("me"),
    content: String::from("a message"),
    reply: false,
    retweet: false,
  };

  assert_eq!("me: a message", tweet.summarize());
}
\end{rust}

最后,让我们实现NewsArticle的特征:

\begin{rust}
impl Summary for NewsArticle {
  fn summarize(&self) -> String {
    format!("{}, by {} ({})", self.headline, self.author, self.location)
  }
}
#[test]
  fn summarize_news_article() {
    let news_article = NewsArticle {
      headline: String::from("Big News"),
      location: String::from("Undisclosed"),
      author: String::from("Me"),
      content: String::from("Big News here, must follow"),
    };

    assert_eq!("Big News, by Me (Undisclosed)", news_article.summarize());
  }
\end{rust}

Rust的特质系统远不止于此,它具备更强大的能力:

\begin{itemize}
\item 
可提供默认方法实现

\item 
可限定参数类型需实现特定特质(或特质组合)

\item 
可泛化地为多种类型实现特质

\item 
融合了面向对象接口、C\#扩展方法和C++概念的多重特性
\end{itemize}

虽然这些高级用法已超出本章讨论范围。最值得铭记的是:Rust处理继承的方式与C++有着根本性差异。

\mySubsubsection{12.3.8}{所有权模型}

Rust有个极具特色的设计,也是其最广为人知的特性——所有权模型。这是Rust针对C++内存安全问题的解决方案,不同于Java/C\#采用垃圾回收机制,Rust通过更明确的内存所有权来解决问题。

正如《Rust编程语言》所述(\url{https://doc.rust-lang.org/book/ch04-01-what-is-ownership.html}):"内存通过所有权系统进行管理,编译器会检查一组规则。任何规则被违反时,程序都无法通过编译。所有权的所有特性都不会在运行时拖慢程序速度。"

Rust所有权的三大核心规则:

\begin{itemize}
\item 
Rust中的每个值都有唯一所有者

\item 
同一时间只能有一个所有者

\item 
当所有者离开作用域时,值将被自动释放
\end{itemize}

先看一个与C++行为相同的例子。对于栈上分配的变量(如整型),其复制行为符合常规认知:

\begin{rust}
#[test]
fn copy_on_stack() {
  let stack_value = 1;
  let copied_stack_value = stack_value;

  assert_eq!(1, stack_value);
  assert_eq!(1, copied_stack_value);
}
\end{rust}

正如预期的那样,这两个变量具有相同的值。但是,如果我们尝试使用在堆上分配的变量写相同的代码,则会收到错误:

\begin{rust}
#[test]
fn copy_on_heap() {
  let heap_value = String::from("A string");
  let copied_heap_value = heap_value;

  assert_eq!(String::from("A string"), heap_value);
  assert_eq!(String::from("A string"), copied_heap_value);
}
\end{rust}

运行这段代码时,我们会遇到错误:

\verb|[E0382]: borrow of moved value: 'heap_value'|

究竟发生了什么?当我们将heap\_value赋值给copied\_heap\_value时,原变量heap\_value随即失效——这与C++的移动语义如出一辙,只是无需开发者额外操作。其底层机制通过两个特质实现:Copy和Drop。若类型实现Copy特质,则表现如第一个示例;若实现Drop特质,则表现如第二个示例。注意:任何类型都不能同时实现这两个特质。

要让上述示例正常工作,我们需要改用克隆机制而非默认的移动语义:

\begin{rust}
#[test]
fn clone_on_heap() {
  let heap_value = String::from("A string");
  let copied_heap_value = heap_value.clone();

  assert_eq!(String::from("A string"), heap_value);
  assert_eq!(String::from("A string"), copied_heap_value);
}
\end{rust}

这个示例可以正常运行,因为此时执行的是值克隆(而非引用同一内存地址),这意味着在堆上进行了新的内存分配。

移动语义在函数调用中同样适用。让我们初始化一个值并将其传递给函数(该函数原样返回此值),观察会发生什么:

\begin{rust}
fn call_me(value: String) -> String {
  return value;
}

#[test]
fn move_semantics_method_call() {
  let heap_value = String::from("A string");

  let result = call_me(heap_value);

  assert_eq!(String::from("A string"), heap_value);
  assert_eq!(String::from("A string"), result);
}
\end{rust}

在编译这段代码时,我们会遇到与之前相同的错误:\verb|error[E0382]: borrow of moved value: 'heap_value'|。这是因为堆上创建的值被移动到call\_me函数中,导致其脱离了当前作用域而被释放。

要使代码正常工作,我们需要指定函数应该借用(borrow)所有权而非取得所有权——这通过引用和解引用操作符实现(与C++的语法相同):

\begin{rust}
fn i_borrow(value: &String) -> &String {
  return value;
}

#[test]
fn borrow_method_call() {
  let heap_value = String::from("A string");

  let result = i_borrow(&heap_value);

  assert_eq!(String::from("A string"), heap_value);
  assert_eq!(String::from("A string"), *result);
}
\end{rust}

Rust引用与C++引用的关键区别在于:Rust引用默认是不可变的(immutable)。

当然,Rust的所有权模型还有更多精妙之处需要探索,但相信这些示例已足够让您初步理解:其运作机制,以及如何有效预防内存安全问题。