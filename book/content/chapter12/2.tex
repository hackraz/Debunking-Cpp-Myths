2001年,当我作为一名初级C++程序员在巴黎工作时,最大的挑战就是让代码实现预期功能。当时开发的是一个工业印刷机的知识库系统,用于帮助操作员识别印刷故障源头。在那个年代,这类桌面应用的主要选择就是Windows平台下的C++,使用Visual C++、微软基础类库(MFC)和Windows API开发,采用微软推崇的文档-视图架构(算是MVC模式的简化版)。这个项目让我备受挑战:不仅要应对C++内存管理的难题,还得处理MFC和Windows API的各种怪癖。当时能依靠的只有官方文档、CodeProject网站(\url{https://codeproject.com})和一位难得有空指导的资深同事。简而言之,我独自面对着一套复杂的技术体系,几乎孤立无援——这就是千禧年初的软件开发现状!不过别误会,我并非抱怨:正是这些挑战让这段经历弥足珍贵,成为我成长的关键。

那时我的注意力完全集中在手头的技术上。虽然听说过PHP,也曾用Java开发过小程序和网页应用,但C++、MFC和Windows API已经耗尽了我的全部精力。每天90分钟的通勤时间倒是让我在一年内读完了整本《指环王》——这大概算是唯一的调剂。

职业生涯第二个重要项目则截然不同:虽然仍是C++开发,但采用了高度结构化的导师制模式,打造了一个当时还未被命名为"NoSQL"的数据库引擎。这段经历让我学会了编写测试用例(当时还没有现成的C++测试框架,我们甚至自建了测试引擎),通过撰写设计文档并与同事评审来掌握软件设计要领,还首次体验了代码审查制度。通过精读Scott Meyers的《Effective C++》《More Effective C++》和Andrei Alexandrescu的《Modern C++ Design》等经典著作,我对C++的理解达到了新高度。

随着C\#的出现,我决定转向新技术。在具备Java和C++基础后,系统学习C\#让我领悟到两点:技术转型会随着经验积累变得越来越容易;每种技术都有其优劣。用C\#开发桌面应用确实更轻松——无需再为内存管理问题提心吊胆,开发效率显著提升,编程过程也更有乐趣。当然,这种便利的代价是牺牲了对底层的控制力,以及某种程度上编程严谨性的降低。

后来我开始思考编程语言泛滥的现象。依我看,纯粹从技术角度出发,我们只需要5-7种语言:网页开发、系统编程、脚本处理各一种,再加上几种面向AI、工作流、方程求解等细分领域的语言。就算我判断有误,20种也绰绰有余。但现实是,如今我们拥有数百种编程语言——从主流语言到小众甚至极客向的Brainfuck、Whitespace等,在TIOBE编程语言排行榜上可见一斑。为何会出现这种现象?

在我看来,这种现象的根源不在于技术需求,而在于文化因素。当然,技术特性确实重要——面向对象和函数式编程特性后来都被引入各种主流语言;安全性、并行与并发、易用性、社区生态等都是编程语言的关键考量。但创造新语言的决策终究源于人,而语言设计中的种种选择又折射出设计者的个人偏好。

文学与哲学的发展轨迹也遵循类似规律:总是主流与反主流思潮相互激荡。浪漫主义是对古典主义的反叛,现实主义又是对浪漫主义的修正。编程语言的演进同样如此:Java是对C++的反拨,Ruby on Rails又是对Java的革新。不同的是,文学潮流受社会变迁影响,而技术浪潮则由双重力量塑造:一是技术生态的变革(如互联网兴起推动Java成为Web开发对C++的替代方案,如今计算向客户端迁移又催生需要C++/Rust等底层语言的WebAssembly应用),二是新一代程序员的审美取向(如Ruby on Rails以自由表达对抗Java的刻板印象,尽管这种"编程愉悦感"缺乏扎实的技术依据——但技术本就不是人类决策的全部要素,开发者亦然)。

至此您应该明白其中的关联:Rust正是对C++的全面回应。它既针对C++现存的技术痛点,也挑战C++的底层方法论。接下来,让我们具体审视Rust带来的变革。













