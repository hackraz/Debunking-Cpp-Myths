
在结束本章前,还有两个重要细节需要说明。首先是C++中函数参数的求值顺序是未指定的(unspecified)。这意味着当调用多参数函数时,编译器可以自由选择任意顺序对参数进行求值。如果参数带有副作用(例如会修改变量值),就可能导致意外结果。

请看以下示例程序:

\begin{cpp}
#include <iostream>

int f (int a, int b, int c) {
  std::cout << "a="<<a<<" b="<<b<<" c="<<c<<std::endl;
  return a+b+c;
}

int main() {
  int i = 1;
  std::cout<<"f="<<f(i++, i++, i++)<<std::endl<<"i="<<i<<std::endl;
}
\end{cpp}

无论你认为这个程序的输出结果是什么,你的答案都可能是错的。

原因正如前文所述:函数参数的求值顺序是未指定的(unspecified)。你可能会问:为什么要这样设计?这背后有着更复杂的历史原因。但在深入探讨之前,让我们先看看不同编译器通过gcc.godbolt.org等平台提供的输出结果:

% Please add the following required packages to your document preamble:
% \usepackage{longtable}
% Note: It may be necessary to compile the document several times to get a multi-page table to line up properly
\begin{longtable}{|l|l|}
\hline
\textbf{Compiler}      & \textbf{Output}                                                         \\ \hline
\endfirsthead
%
\multicolumn{2}{c}%
{{\bfseries Table \thetable\ continued from previous page}} \\
\endhead
%
Microsoft Visual C++ (after 2005) & \begin{tabular}[c]{@{}l@{}}a=1 b=1 c=1 \\ \\ f=3 \\ \\ i=4\end{tabular} \\ \hline
Microsoft VS.NET 2003  & \begin{tabular}[c]{@{}l@{}}a=3 b=2 c=1 \\ \\ f=6 \\ \\ i=4\end{tabular} \\ \hline
Microsoft Visual C++ 6 & \begin{tabular}[c]{@{}l@{}}a=1 b=1 c=1 \\ \\ f=3 \\ \\ i=4\end{tabular} \\ \hline
ICC and Clang agree on this…      & \begin{tabular}[c]{@{}l@{}}f=a=1 b=2 c=3 \\ \\ 6 \\ \\ i=4\end{tabular} \\ \hline
GCC, after 6.5         & \begin{tabular}[c]{@{}l@{}}f=a=3 b=2 c=1 \\ \\ 6 \\ \\ i=4\end{tabular} \\ \hline
GCC, before 6.5        & \begin{tabular}[c]{@{}l@{}}a=3 b=2 c=1 f=\\ \\ 6 \\ \\ i=4\end{tabular} \\ \hline
Turbo C Lite and Borland C++55    & \begin{tabular}[c]{@{}l@{}}a=3 b=2 c=1 \\ \\ f=6 \\ \\ i=1\end{tabular} \\ \hline
\end{longtable}


因此,我们面临着多种可能的选择——有些较为直观,有些则相当特殊。所有这些看似奇怪的结果都宣称自己是"正统答案",即便同一厂商的不同编译器版本也会给出不同结果。而它们确实都是正确的。

简而言之,允许编译器自由选择求值顺序,使其能够进行我们开发者可能注意不到的性能优化:

\begin{itemize}
\item 
编译器可以重排指令以利用CPU流水线

\item 
最小化寄存器使用

\item 
提升缓存效率,并指定严格顺序会限制这些优化机会
\end{itemize}

不同硬件架构可能需要不同的最优求值策略。不指定求值顺序使C++代码能更容易适配多种架构,而无需修改代码本身。

此外,不指定求值顺序也保持了C++语言规范的简洁性。若为所有表达式指定严格顺序,将增加语言定义的复杂性,加重编译器开发者的负担。考虑到现行标准已近2000页,或许没必要再增加数百页来详细规定参数求值的复杂性。

不过,本节开头承诺要提到的第二点是:虽然运算符优先级和结合性决定了表达式的分组和解析方式,但它们并不控制求值顺序。这意味着即使你知道表达式将如何分组,其各部分实际的求值顺序仍可能变化。

请看以下简短示例程序:

\begin{cpp}
#include <iostream>

int main() {
  int i = 4;
  i = ++i + i++;
  std::cout << i << std::endl;
  return 0;
}
\end{cpp}

这段代码极其简短——简直不能再短了——但包含了一些相当棘手的代码,特别是++i + i++这个表达式。这段代码如此棘手,以至于不同编译器对其执行顺序无法达成一致。

有些编译器选择先执行++i(使i变为5,并作为加法的左操作数),然后执行i++(此时会使用已递增的i值5,再将其递增到6,但由于后置递增的特性,加法的右操作数仍使用值5),最后将这个结果赋值给i。因此得到5 + 5 = 10。

而另一些编译器则决定先执行i++,于是加法的右操作数保持为4,同时将i递增到5。接着执行++i,此时看到i值已经是5,决定使用它并再递增到6,于是加法的左操作数为6。这样得到6 + 4 = 10。

现在回想起来,不指定求值顺序实际上促使了解这一特性的开发者编写不依赖特定求值顺序的代码。这有助于写出更健壮、可移植性更好的代码,因为开发者必须避免对求值顺序产生隐含依赖。因此,针对上述情况,正确的修改方式应该是类似下面这样的代码:

\begin{cpp}
#include <iostream>

int main() {
  int i = 4;
  int preIncrement = ++i; // i is now 5
  int postIncrement = i++; //postIncrement is 5, i is now 6
  i = preIncrement + postIncrement;
  std::cout << i << std::endl; // Output will be 10
  return 0;
}
\end{cpp}

虽然这种情况可能比较罕见(因为上述代码有点人为设计的痕迹),但它确实是个问题——特别是当我们遇到类似以下场景时:

\begin{cpp}
int f() { std::cout << "f"; return 1; }
int g() { std::cout << "g"; return 2; }
int result = f() + g();
\end{cpp}

无论编译器如何决定两个函数的调用顺序,result的值最终都会是3,但输出结果可能是"fg"或"gf"。

考虑到所有这些因素,我们或许以为自己已经完全掌握了C++中的顺序规则。尽管本章尝试涵盖了所有可能的隐含情况,但我们不能保证你不会遇到任何"不按顺序出牌"的情形。C++作为一门范围极广且语法独特的语言,如果有人存心为之,确实可能踩到某些编译器的"痛脚"。

