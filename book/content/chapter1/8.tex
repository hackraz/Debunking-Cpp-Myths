在本章中,我们研究了一句话:C++ 非常难学。那么,是吗? 

我们研究了 C++ 的历史,以及最初编写最简单的程序确实是一个挑战。我们了解了 Java、 C\# 和 Python 如何处理程序员在 C++ 中面临的一些问题,以及 C++ 标准在过去 15 年中如何乎意料地快速发展以消除其障碍。

虽然您现在可以编写类似于 Java 或 C\# 的 C++ 代码,但您可能仍然需要了解内存管理,我使用移动语义来举例说明这一事实。我们还看到,学习 C++ 的方法随着语言和时代的发而发展,Stroustrup 只是顺便引入指针,并迅速切换到 STL 中可用的更高级别结构。我们到,修改后的 TDD 循环可以帮助人们以探索性的方式学习 C++,而不会被错误消息和语的复杂性所淹没。

我们还指出,C++ 在工具和可移植性方面有一个缺点。在 C++ 中,安装新的依赖项是一整事,这与 Java、Python 或 C\# 不同,它们提供了一个事实上的标准命令来管理软件包。这能会让那些想成为 C++ 程序员的人望而却步,他们不会进行更深入的分析。

最后,尽管标准取得了进展,但我们不能忘记世界上 C++ 代码的庞大规模,并且没有达到新的标准。很有可能,即使你学习了现代 C++,你的工作迟早也会涉及处理旧代码。

因此,我们得出结论,C++ 仍然比 Java、C\# 或 Python 更难学习,但它比以往任何时候都更近,并且该语言的强大功能仍然对一部分程序员具有吸引力。

在下一章中,Ferenc 将研究这个问题:每个 C++ 程序都是标准的吗?或者,也许程序员的动力是解决问题并选择最适合他们环境的解决方案,忽略标准,甚至创造出一段时间后终出现在标准中的习语。