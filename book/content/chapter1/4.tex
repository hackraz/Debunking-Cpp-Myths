尽管 C++ 标准一直在朝着更简洁、现代的方向演进,但许多学习资料却依然停留在过去。我理解要跟上 C++ 标准的变化并不容易,尤其是在 2010 年之后,语言更新的步伐明显加快。一个常见的问题是:在实际开发中,究竟有多少代码真正采用了最新的标准?学生是否仍然需要掌握那些旧版 C++ 的写法,以便应对那些已有数十年历史的代码库?

虽然这种情况确实存在,但我们终究必须向前迈进,而 C++ 的设计者 Bjarne Stroustrup 也持这一观点。他撰写的书《C++ 程序设计原理与实践》(Programming: Principles and Practice Using C++)在 2024 年推出的第三版中,针对面向编程初学者系统地介绍了 C++,是一本非常优秀的入门教材。书中配有大量实用的教学示例和课件,非常适合教师授课或自学使用。需要注意的是,Stroustrup 并没有回避指针和内存管理这些传统难点,而是以一种循序渐进的方式讲解了最基本的知识,并迅速引导读者了解如何在现代 C++ 中避免直接使用它们。

以第 16 章相关的教学幻灯片为例,该章节聚焦于数组相关的内容。从讲解原始数组(naked arrays)开始,解释了数组与指针之间的关系,以及使用指针时可能遇到的问题。随后,课程逐步引入一系列更安全、更现代的替代方案:vector、set、map、unordered\_map、array、string、unique\_ptr、shared\_ptr、span 和 not\_null指针等。这组幻灯片最后通过一个判断回文字符串的示例收尾,展示了多种实现方式,并比较了它们在代码安全性与简洁性方面的差异。因此,这一章节的核心目标是揭示数组与指针带来的种种问题,并展示标准模板库(STL)结构如何规避这些问题。

最终写出的代码风格已经非常接近 Java 或 C\# 的写法,但 Stroustrup 也指出,指针运算在某些数据结构的底层实现中仍然具有价值。换句话说,它应在确实需要进行高度性能优化时谨慎使用。由此可见,C++ 的设计者并不是在回避指针和内存管理,而是在努力减少与之相关的潜在风险。相比 C++98 时代,现代 C++ 让开发者能够将更多精力放在业务逻辑上,而不是频繁处理内存细节 --- 仍需比 Java 或 C\# 开发者投入更多的关注。

那么问题依然存在:初学者是否可以在不深入理解指针的情况下学习 C++?另一种教学方法似乎表明这是可能的 --- 前提是我们希望培养的是库的使用者,而不是库的设计者。