尽管C++标准一直在朝着更简洁的方向发展,但许多学习资料却依然停留在过去。我能理解要跟上C++标准的变化并不容易,尤其是在2010年之后它更新的速度明显加快。一个常常存在的问题是:有多少代码真正使用了最新的标准?学生是否无论如何都得学习旧版C++的方法,以便处理那些已有数十年历史的代码库呢?

尽管存在这种可能性,但我们终究必须向前迈进,而Bjarne Stroustrup也持同样的观点。他撰写的书籍《C++程序设计原理与实践》(Programming: Principles and Practice Using C++)于2024年出版的第三版,是面向编程初学者的,引导读者全面了解C++语言。这本书是对C++非常好的入门介绍,并配有实用的教学示例和课件,非常适合想要教授或学习这门语言的人。值得注意的是,Stroustrup并没有回避指针和内存管理这一主题,而是讲解了最基本的必要知识,并立即展示了现代C++中如何避免使用它们的方式。

让我们以第16章相关的幻灯片为例,这些幻灯片的重点是数组。它们从讲解原始数组(naked arrays)开始,解释了它们与指针之间的联系,以及在使用指针时可能会遇到的问题。随后介绍了一系列替代方案:vector、set、map、unordered\_map、array、string、unique\_ptr、shared\_ptr、span 和 not\_null。这套幻灯片最后通过一个回文字符串判断的示例收尾,该示例展示了多种不同的实现方式,并比较了它们在代码安全性与简洁性方面的差异。因此,本章的主要目的就是展示数组和指针所带来的各种问题,以及标准模板库(STL)结构如何帮助我们避免这些问题。

最终写出的代码非常类似于 Java 或 C\# 中的实现方式。不过,Stroustrup 指出,指针运算在实现数据结构时仍然有用。换句话说,应该谨慎地使用它,仅当你确实需要进行高度优化时才使用。因此我们可以得出结论:这门语言的设计者并不回避指针和内存管理,而是专注于消除许多与之相关联的潜在问题。这让 C++ 程序员相比 C++98 时代可以更少地关注内存管理,虽然仍需比使用 Java 或 C\# 的程序员多操心一些。

问题依然存在:初学者是否可以在不深入思考指针的情况下学习 C++?另一种教学方法似乎证明这是可能的——前提是我们希望培养的是库的使用者,而不是库的设计者。