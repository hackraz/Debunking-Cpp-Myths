C++ 真的和 Java、C\#、PHP、JavaScript 或 Python 一样容易了吗?现在,这门语言经历了多次现代化改进,但答案依然是否定的。但更重要的是:\textbf{C++ 应该像这些语言一样容易学习吗?}

事实上,C++ 的“消亡”已经被预言了很长一段时间。Java、随后的 C\#,以及如今的 Rust 都曾为对这位备受争议的“老将”的全面替代品。但现实是,它们各自开辟了自己的生态领域,而 C++ 依旧在那些需要精细优化或运行于资源受限环境的系统中占据主导地位。此外,全球范围内仍有数以百万计行的 C++ 代码仍在使用中,其中一些甚至已有几十年历史。虽然其中一部分可以通过重构转向云原生、无服务器架构或微服务模式,但仍有许多问题更适合用 C++ 所代表的工程风格来解决 --- 强调性能、控制力与可移植性。

因此我们可以得出结论:C++ 在开发世界中有着不可替代的独特地位。任何一门新语言若想完全取代它,都将面临一场艰难的挑战。这也带来了一个自然的结果:C++ 中某些特定的部分,注定会比其他语言难掌握。

Java 或 C\# 可以屏蔽你对内存分配机制的思考,当把参数传递给另一个方法时,内存会发生什么变化;而 C++ 必须正视这些问题,并赋予开发者根据上下文进行优化的能力。

因此,如果想真正理解 C++,就无法回避内存管理这个核心议题。幸运的是,这项任务已经远不像曾今那样困难了。

让我们通过比较不同语言如何处理内存分配与释放,来深入分析它们之间的差异。

Java 使用了一种完全面向对象的设计方式,其中每一个值都是一个对象。C\# 的设计者则采用了值类型与引用类型并存的方式:值类型包括常见的数值、字符、结构体和枚举,而引用类型则对应对象。Python 中,每个值也都是对象,其类型可以在程序运行过程中动态确定。这三种语言都依赖垃圾回收机制来自动管理内存的释放。Python 还引入了引用计数机制,因此在某些场景下,开发者甚至可以选择性地禁用垃圾回收机制。

而在 C++98 标准中,语言本身没有提供任何内建的指针释放机制,而是将内存管理的全部权力和责任交到了开发者手中。这种设计虽然带来了极高的灵活性,但也导致了许多问题。例如,初始化了一个指针并为其分配了一块较大的内存空间,然后将这个指针传递给了其他函数,那么谁应该负责释放这块内存呢?

请看下面这段简单的代码示例:

\begin{cpp}
BigDataStructure* pData = new pData();
call1(pData);
call2(pData);
call3(pData);
\end{cpp}

调用者是否应该负责释放 pData 中分配的内存?是 call3 来做这件事吗?如果 call3 又将同一个 pData 实例传递给了另一个函数呢?谁最终应当承担释放内存的责任?若 call2 执行失败,又会发生什么?

由此可见,内存释放的责任并不总是明确的。因此,我们必须为每一个函数、或者更准确地说,每一个作用域,明确定义谁来负责释放内存。随着程序逻辑和数据流复杂度的增加,这个问题的复杂性也随之上升。大多数使用其他主流语言的开发者面对这种情况时,要么感到困惑,要么干脆选择忽略内存管理责任,最终导致内存泄漏,或访问已经释放的内存区域 --- 这些正是 C++98 时代常见的问题。

Java、C\# 和 Python 则通过完全免除开发者手动处理内存的方式,解决了所有这些问题。它们依赖于两种关键技术:引用计数 和 垃圾回收机制。

引用计数的工作原理如下:每当一个值被复制时,其引用计数就递增;当变量离开其作用域时,引用计数递减;当引用计数变为零时,系统便自动释放对应的内存空间。

垃圾回收器则与之类似,但它是定期运行的,并能够检测循环引用等复杂情况,从而确保即使结构复杂的内存也能被正确释放,尽管可能带有一定的延迟。

事实上早在 2000 年代初,我们就可以在 C++ 中实现类似的引用计数机制。这种设计模式被称为 智能指针,大大减少了开发者对底层内存管理的关注。

早在 C++ 的初始版本中,语言本身就提供了另一种更优雅的方式来应对这一问题:按引用传递。这也是为什么 Java、C\# 和 Python 都将对象默认以引用方式传递的原因:这是一种非常自然且方便的做法。

可以创建一个对象并为其分配内存,然后按引用传递这个对象 --- 最棒的是,当它离开作用域时,内存会自动释放,无需手动干预。

下面来看一个与使用原始指针相似的示例:

\begin{cpp}
BigDataStructure data{};

call1(data);

call2(data);

call3(data);

...

void call1(BigDataStructure& data){
  ...
}
\end{cpp}

这一次,call1 中具体发生了什么已不再重要;只要离开了 data 初始化所在的作用域,内存就会自动释放。引用类型的唯一限制在于:它所绑定的变量无法重新分配新的内存地址。就我个人而言,这其实是一个巨大的优势,当允许修改底层内存分配,事情往往会迅速变得复杂起来。事实上,我更倾向于使用 const\& 来传递每一个值。当然,也确实存在一些需要通过内存重新分配来实现的、高度优化的多态数据结构,但它们的应用场景非常有限。

再来看前面的这段程序 --- 如果忽略掉 call1 中的 \& 符号,并将函数重命名为符合 Java 或 C\# 命名风格的形式,则这段代码与 Java 或 C\# 的写法无异。因此从一开始,C++ 就具备了与其他主流语言相似的潜力。那为什么如今它门显得那样的“格格不入”呢?

在 C++ 中始终无法完全回避内存管理。对 Java 或 C\# 开发者来说,前面这段代码可能不会引发太多思考,但C++不同。标准化委员会也意识到了这一点:在某些情况下,我们需要在一个函数中分配内存,而在另一个函数中释放它。理想情况下,应避免直接使用原始指针来完成这项任务,所以移动语义应运而生。

\begin{myNotic}{Note}
移动语义是 C++11 引入的一项关键特性,旨在通过消除不必要的对象拷贝来提升程序性能。允许将资源(如动态内存、文件句柄等)从一个对象“转移”到另一个对象,而非复制。这对于管理资源的对象尤其有益。

要使用移动语义,需要为类实现两个关键组件:

\begin{itemize}
\item 
移动构造函数,用于将资源从一个右值(临时对象)转移到新创建的对象;

\item 
移动赋值运算符,用于将资源从一个右值转移到一个已存在的对象。
\end{itemize}

此外,标准库提供的工具函数 std::move 可以将一个对象转换为右值引用,从而启用移动语义。在某些条件下,编译器会自动生成默认的移动构造函数和移动赋值运算符。
\end{myNotic}

下面的示例中,请参阅如何使用 move 语义,将变量的范围移动到函数进程:

\begin{cpp}
BigDataStructure data{};

process(data);

...
void process(BigDataStructure&& data){
}
\end{cpp}

从表面上看,仅仅多出了两个“\&”符号而已,但其背后的行为却截然不同。变量 data 的作用域被“移动”到了被调用的函数中,随后传入 process,而内存会在离开该作用域时自动释放。

移动语义 使得我们能够避免复制大型数据对象,并将内存管理的责任一并转移给接收它的函数或对象。这是目前我们所讨论的语言中,C++ 所独有的一项机制。据我所知,实现了类似机制的编程语言,是另外两门专注于系统级编程的语言:Rust 和 Swift。

尽管如今的 C++ 在语法层面越来越像 Java 或 C\#,但仍然要求开发者对内存分配与释放机制有更深入的理解。我们或许已经不再需要纠结那些曾让人头疼的语法细节题,但要真正掌握 C++,仍需学习比其他语言更多的底层知识。

内存管理虽然是学习过程中的一个主要难点,但它绝不是让 C++ 更难掌握的唯一因素。还有一些语言特性,对于新手来说可能略显陌生,甚至有些令人困扰:

\begin{itemize}
\item 
需要使用 \#ifndef 预处理指令(或非标准但主流支持的 \#pragma once)来确保头文件只被包含一次,避免重复定义带来的编译错误。

\item 
需要维护单独的 .h 头文件,并遵循 .h 与 .cpp 文件之间的各种约定或规则,这种分离在其他现代语言中早已不复存在,对新手来说可能显得繁琐且不够直观。

\item 
使用 virtual void methodName() = 0 的方式来定义接口,这种语法不仅冗长,而且对于初次接触面向对象编程的人来说显得非常奇怪和不够自然。
\end{itemize}

尽管现代 IDE 可以自动处理许多规则和指南,帮助我们正确使用这些复杂的机制,但它们的存在本身仍引发了一个值得思考的问题:我们真的还需要它们吗?

然而,还有一个更加难以忽视的事实:在 C++ 中,构建程序并引入外部依赖并不是一件简单直接的事情。尽管 Java 也有自身的缺点,但它拥有统一的编译器,以及像 Maven 和 Gradle 这样标准化的依赖管理工具,开发者只需一个简单的命令就能下载并集成新的库。C\# 虽然在过去也面临类似的挑战,但如今社区广泛使用的 NuGet 已基本成为获取外部库的标准方式。Python 更是凭借 pip 提供了极为便捷的包管理体验。

相比之下,C++ 的开发者则需要付出更多努力。不同于 Java 和 C\# 依赖虚拟机运行,C++ 程序必须为每一个目标平台单独编译,并且所依赖的库也必须与该平台匹配。当然,我们也有一些可用的工具,最常提及的包管理器包括 Conan 和 vcpkg,而在构建系统方面,CMake 几乎已成为事实上的主流选择。问题在于:这些工具中没有一个是官方标准。虽然 Java 的 Maven/Gradle 和 C\# 的 NuGet 最初也不是官方标准,但由于开发工具迅速集成并广泛采用,逐渐演变为行业标准。而 C++ 在这方面还有很长的路要走。我们将在后续章节中专门讨论这些问题。但不可否认的是,在尝试运行一些简单程序时所面临的复杂性,正是造成 C++ 学习门槛较高的另一个重要因素。

至此,我们已经比较了 C++ 相较于其他语言在多个方面的复杂之处,也可以看出,尽管 C++ 已经比过去更容易使用,但它依然不像 Java 或 C\# 那样直观和简便。

那么,回到最初的问题:C++ 真的很难学吗?

为了进一步探讨这个问题,让我们来看看初学者在学习 C++ 时通常可以采用的三种不同路径。


