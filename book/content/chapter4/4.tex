在本章中,作者尝试对 Linux 和 Windows 上的应用程序启动过程提供一个并不那么全面的概述。我们介绍了程序执行初期的一些关键步骤,包括 main() 函数之前的初始化流程。虽然这些内容远未达到这两个平台本身所要求的完整程度,但若真要详尽覆盖这个主题,本书的书名恐怕也得改写成别的了——因为这是一个庞大而小众的主题,并不会吸引广泛的开发者群体。

通过探索 Linux 中的 ELF 格式、理解 execve() 系统调用以及分析 \_start() 函数,你已经获得了关于底层架构和初始化例程的宝贵知识。同样地,在对 Windows 的讨论中,我们重点介绍了控制台应用程序与 GUI 应用程序的启动流程,强调了各个节区(section)的作用,以及它们是如何共同协作来启动你的那个“总是不听话”的程序的,尤其是在它无法正常运行时。

为了给大家留下进一步深入研究的空间,我们建议你进行动手实验:创建并分析二进制文件、修改启动例程、观察其在不同操作系统上的行为变化。你甚至可以手动修改可执行文件头中的各种地址,来看看会发生什么、以及它们是怎么崩溃的。

这种实践性的方法不仅会强化你在本章中学到的概念,还会让你更深入、更实际地掌握应用程序的启动机制。通过积极的探索和实验,你将提升自己在软件开发领域中排错、优化和创新的能力,同时还能学到许多关于软件及其运行环境既实用又有趣的冷知识。

在下一章中,将介绍类成员声明的正确顺序问题。通过一位开发者踏上编写无 bug 代码之旅的冒险故事,我们将一起发现那些隐藏的 bug。

请继续阅读吧,你一定会有所收获的。