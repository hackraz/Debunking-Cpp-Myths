
当我们刚开始在学校或大学学习 C++ 时,在第一门 C++ 课程上,老师会告诉我们:“亲爱的同学们,这是主函数:void main(void)。你的程序就是从这里开始运行的。” 就是这样。

本章到此结束——翻过这一页,我们下一章再见。

然而,这句话并不正确。我之所以写下 void main(void),只是为了唤起你的好奇心,让你保持警觉的状态。事实上,对于所有已经有一定经验的 C++ 程序员来说,他们应该都知道,void main(void) 距离标准 C++ 的规范,就像“尼莫点”(地球上距离最近陆地最远的海洋地点)距离最近的陆地一样遥远。

哦,你还在这里!这意味着你确实读了这些小字说明。太好了——我们程序员就该始终关注细节,比如我们的应用程序是如何被底层操作系统加载并执行到内存中的。

我们生活在一个自由的世界中,因此我们可以根据自己的意愿选择多种操作系统。因此,我们将介绍应用程序在 Linux 和 Windows 下是如何被加载和执行的。

这两个操作系统在加载和执行编译后的二进制文件方面存在显著差异。其中一种系统(不难猜出是哪一个)允许我们追踪这个奇特过程的所有代码路径,直至深入到底层内核的最底层;而对另一种系统,我们必须依赖现有的文档、书籍以及各种信息来源,这些资料需要热衷于底层技术的研究者自行收集整理。

由于 Linux 处理这一操作的方式与 BSD 家族的操作系统(如 FreeBSD、NetBSD 等)非常相似,因此在接下来的讨论中,我们会避免频繁提及这些系统。为了让你在追求知识的过程中也能感到有趣,我们仍然希望提供最新的信息,因此我们决定不为那些现在已经不再活跃使用的特殊操作系统(例如 MS-DOS)提供相关内容,除非你恰好在 Deutsche Bahn 工作\footnote{\url{https://www.theregister.com/2024/01/30/windows_311_trundles_on/}}。

但在我们深入探讨之前,让我们先展示一下本章将用于举例说明上述功能的测试程序:

\begin{cpp}
#include <cstring>
#include <cstdio>

struct A {
  A(const char* p_a):m_a(new char[32]) { strcpy(m_a, p_a);
    printf("A::A : %s\n", p_a);
  }

  ~A() {

  printf("A::~A : %s\n", m_a);
    delete[] m_a;
  }

  volatile const char* get() const {return m_a;}
private:
  char* m_a;
};

const char* my_string= "Hello string";
A my_a(my_string);

const char* my_other_string = "Go away string";
A my_other_a(my_other_string);

int main() {
  printf("Hello, World, %s, %s\n", my_a.get(), my_other_a.get()); 
}
\end{cpp}

在符合标准的系统上编译和运行时,前面的应用程序将生成以下输出,正如符合标准的程员所期望的那样:

\begin{shell}
A::A : Hello string
A::A : Go away string
Hello, World, Hello string, Go away string
A::~A : Go away string
A::~A : Hello string
\end{shell}

是的,我们有意没有使用 cout 和其他流操作,因为我们要让程序保持简单。

我们不希望生成的代码被干扰,因为我们计划深入研究编译后的可执行文件。

另外,请注意,这段代码是为了本章专门编写的示例代码,是一种合成代码。作者完全清楚使用 strcpy 可能带来的内存溢出错误,因此建议读者“听作者的话,而不是照着作者做”——请“不要使用 strcpy”。

回到我们的最初目标,让我们来看看操作系统是如何加载并执行应用程序的。

亲爱的读者,如果你觉得下面的讨论层次太低,不太感兴趣,请记住:C++ 程序被编译为原生代码,并以底层操作系统所能提供的最高速度运行。

鉴于这一点,我们认为每一位 C++ 程序员都有必要了解操作系统是如何处理他们的代码的,以及编译器在处理完源文件并“吐出”一个可执行文件之后,究竟发生了什么。我们将尽量避免过于底层的细节,只展示真正必要的内容,以便你能够充分理解这一过程的重要性。











































