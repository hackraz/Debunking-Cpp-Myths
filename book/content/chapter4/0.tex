
\begin{flushright}
\textit{--- main之前事存于main}
\end{flushright}

对于在多种操作系统上使用 C++ 的开发者而言,理解应用程序的入口点,及其背后的底层架构至关重要。它不仅仅是一个简单的起点,而是通往理解程序如何从无到有、启动运行的钥匙。本章将深入剖析一个应用程序的启动过程,特别关注在进入熟悉的 main() 函数之前,那些默默完成初始化工作的底层代码。

首先在 Linux 系统的土壤中探索这一过程,将揭开可执行与可链接格式(Executable and Linkable Format,ELF)的神秘面纱,详细解读 execve() 系统调用如何加载并执行 \_start() 函数。这个 \_start() 函数就像是舞台的幕后总导演,在将控制权交给 main() 之前,精心布置好整个运行时环境。此外,还会触及一些编译器提供的巧妙扩展,允许开发者对这一启动流程进行精细的操控。

随后,将目光转向 Windows 平台,细致地分析其可移植可执行文件格式(Portable Executable,PE)中各个功能各异的节区(sections)。

为了更直观、更深入地洞悉这两个平台上可执行文件的内部构造,我们将借助一款强大的工具 --- Ghidra。它是由美国国家安全局(NSA)开发的优秀开源软件逆向工程套件,不仅能反汇编、反编译和调试二进制文件,更能为我们揭示应用程序启动过程中的底层机制。

完成本章后,你将对以下内容有更深入的理解:

\begin{itemize}
\item 
Linux 和 Windows 下可执行文件的格式与进程启动机制

\item 
如何修改和控制应用程序的启动过程
\end{itemize}

\begin{myTip}{什么是 Ghidra?}
Ghidra 是由美国国家安全局(NSA)开源的软件逆向工程套件。它功能强大,支持分析多种格式和平台的编译代码。通过提供反编译、反汇编和调试二进制文件的工具,Ghidra 让用户能够更轻松地理解和分析机器代码,是探索软件内部运作机制的得力助手。
\end{myTip}





